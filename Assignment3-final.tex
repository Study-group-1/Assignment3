% Options for packages loaded elsewhere
\PassOptionsToPackage{unicode}{hyperref}
\PassOptionsToPackage{hyphens}{url}
%
\documentclass[
]{article}
\usepackage{lmodern}
\usepackage{amsmath}
\usepackage{ifxetex,ifluatex}
\ifnum 0\ifxetex 1\fi\ifluatex 1\fi=0 % if pdftex
  \usepackage[T1]{fontenc}
  \usepackage[utf8]{inputenc}
  \usepackage{textcomp} % provide euro and other symbols
  \usepackage{amssymb}
\else % if luatex or xetex
  \usepackage{unicode-math}
  \defaultfontfeatures{Scale=MatchLowercase}
  \defaultfontfeatures[\rmfamily]{Ligatures=TeX,Scale=1}
\fi
% Use upquote if available, for straight quotes in verbatim environments
\IfFileExists{upquote.sty}{\usepackage{upquote}}{}
\IfFileExists{microtype.sty}{% use microtype if available
  \usepackage[]{microtype}
  \UseMicrotypeSet[protrusion]{basicmath} % disable protrusion for tt fonts
}{}
\makeatletter
\@ifundefined{KOMAClassName}{% if non-KOMA class
  \IfFileExists{parskip.sty}{%
    \usepackage{parskip}
  }{% else
    \setlength{\parindent}{0pt}
    \setlength{\parskip}{6pt plus 2pt minus 1pt}}
}{% if KOMA class
  \KOMAoptions{parskip=half}}
\makeatother
\usepackage{xcolor}
\IfFileExists{xurl.sty}{\usepackage{xurl}}{} % add URL line breaks if available
\IfFileExists{bookmark.sty}{\usepackage{bookmark}}{\usepackage{hyperref}}
\hypersetup{
  pdftitle={Assignment 3 - Applying meta-analytic priors},
  pdfauthor={Riccardo Fusaroli},
  hidelinks,
  pdfcreator={LaTeX via pandoc}}
\urlstyle{same} % disable monospaced font for URLs
\usepackage[margin=1in]{geometry}
\usepackage{color}
\usepackage{fancyvrb}
\newcommand{\VerbBar}{|}
\newcommand{\VERB}{\Verb[commandchars=\\\{\}]}
\DefineVerbatimEnvironment{Highlighting}{Verbatim}{commandchars=\\\{\}}
% Add ',fontsize=\small' for more characters per line
\usepackage{framed}
\definecolor{shadecolor}{RGB}{248,248,248}
\newenvironment{Shaded}{\begin{snugshade}}{\end{snugshade}}
\newcommand{\AlertTok}[1]{\textcolor[rgb]{0.94,0.16,0.16}{#1}}
\newcommand{\AnnotationTok}[1]{\textcolor[rgb]{0.56,0.35,0.01}{\textbf{\textit{#1}}}}
\newcommand{\AttributeTok}[1]{\textcolor[rgb]{0.77,0.63,0.00}{#1}}
\newcommand{\BaseNTok}[1]{\textcolor[rgb]{0.00,0.00,0.81}{#1}}
\newcommand{\BuiltInTok}[1]{#1}
\newcommand{\CharTok}[1]{\textcolor[rgb]{0.31,0.60,0.02}{#1}}
\newcommand{\CommentTok}[1]{\textcolor[rgb]{0.56,0.35,0.01}{\textit{#1}}}
\newcommand{\CommentVarTok}[1]{\textcolor[rgb]{0.56,0.35,0.01}{\textbf{\textit{#1}}}}
\newcommand{\ConstantTok}[1]{\textcolor[rgb]{0.00,0.00,0.00}{#1}}
\newcommand{\ControlFlowTok}[1]{\textcolor[rgb]{0.13,0.29,0.53}{\textbf{#1}}}
\newcommand{\DataTypeTok}[1]{\textcolor[rgb]{0.13,0.29,0.53}{#1}}
\newcommand{\DecValTok}[1]{\textcolor[rgb]{0.00,0.00,0.81}{#1}}
\newcommand{\DocumentationTok}[1]{\textcolor[rgb]{0.56,0.35,0.01}{\textbf{\textit{#1}}}}
\newcommand{\ErrorTok}[1]{\textcolor[rgb]{0.64,0.00,0.00}{\textbf{#1}}}
\newcommand{\ExtensionTok}[1]{#1}
\newcommand{\FloatTok}[1]{\textcolor[rgb]{0.00,0.00,0.81}{#1}}
\newcommand{\FunctionTok}[1]{\textcolor[rgb]{0.00,0.00,0.00}{#1}}
\newcommand{\ImportTok}[1]{#1}
\newcommand{\InformationTok}[1]{\textcolor[rgb]{0.56,0.35,0.01}{\textbf{\textit{#1}}}}
\newcommand{\KeywordTok}[1]{\textcolor[rgb]{0.13,0.29,0.53}{\textbf{#1}}}
\newcommand{\NormalTok}[1]{#1}
\newcommand{\OperatorTok}[1]{\textcolor[rgb]{0.81,0.36,0.00}{\textbf{#1}}}
\newcommand{\OtherTok}[1]{\textcolor[rgb]{0.56,0.35,0.01}{#1}}
\newcommand{\PreprocessorTok}[1]{\textcolor[rgb]{0.56,0.35,0.01}{\textit{#1}}}
\newcommand{\RegionMarkerTok}[1]{#1}
\newcommand{\SpecialCharTok}[1]{\textcolor[rgb]{0.00,0.00,0.00}{#1}}
\newcommand{\SpecialStringTok}[1]{\textcolor[rgb]{0.31,0.60,0.02}{#1}}
\newcommand{\StringTok}[1]{\textcolor[rgb]{0.31,0.60,0.02}{#1}}
\newcommand{\VariableTok}[1]{\textcolor[rgb]{0.00,0.00,0.00}{#1}}
\newcommand{\VerbatimStringTok}[1]{\textcolor[rgb]{0.31,0.60,0.02}{#1}}
\newcommand{\WarningTok}[1]{\textcolor[rgb]{0.56,0.35,0.01}{\textbf{\textit{#1}}}}
\usepackage{graphicx}
\makeatletter
\def\maxwidth{\ifdim\Gin@nat@width>\linewidth\linewidth\else\Gin@nat@width\fi}
\def\maxheight{\ifdim\Gin@nat@height>\textheight\textheight\else\Gin@nat@height\fi}
\makeatother
% Scale images if necessary, so that they will not overflow the page
% margins by default, and it is still possible to overwrite the defaults
% using explicit options in \includegraphics[width, height, ...]{}
\setkeys{Gin}{width=\maxwidth,height=\maxheight,keepaspectratio}
% Set default figure placement to htbp
\makeatletter
\def\fps@figure{htbp}
\makeatother
\setlength{\emergencystretch}{3em} % prevent overfull lines
\providecommand{\tightlist}{%
  \setlength{\itemsep}{0pt}\setlength{\parskip}{0pt}}
\setcounter{secnumdepth}{-\maxdimen} % remove section numbering
\usepackage{booktabs}
\usepackage{longtable}
\usepackage{array}
\usepackage{multirow}
\usepackage{wrapfig}
\usepackage{float}
\usepackage{colortbl}
\usepackage{pdflscape}
\usepackage{tabu}
\usepackage{threeparttable}
\usepackage{threeparttablex}
\usepackage[normalem]{ulem}
\usepackage{makecell}
\usepackage{xcolor}
\ifluatex
  \usepackage{selnolig}  % disable illegal ligatures
\fi

\title{Assignment 3 - Applying meta-analytic priors}
\author{Riccardo Fusaroli}
\date{}

\begin{document}
\maketitle

\hypertarget{assignment-3}{%
\subsection{Assignment 3}\label{assignment-3}}

In this assignment we do the following: - we run a Bayesian
meta-analysis of pitch variability in ASD, based on previously published
literature - we analyze pitch variability in ASD in two new studies
using both a conservative and a meta-analytic prior - we assess the
difference in model quality and estimates using the two priors.

The questions you need to answer are: What are the consequences of using
a meta-analytic prior? Evaluate the models with conservative and
meta-analytic priors. Discuss the effects on estimates. Discuss the
effects on model quality. Discuss the role that meta-analytic priors
should have in scientific practice. Should we systematically use them?
Do they have drawbacks? Should we use them to complement more
conservative approaches? How does the use of meta-analytic priors you
suggest reflect the skeptical and cumulative nature of science?

\hypertarget{step-by-step-suggestions}{%
\subsubsection{Step by step
suggestions}\label{step-by-step-suggestions}}

\#\#\#\#Step 1: Perform a meta-analysis of pitch variability from
previous studies of voice in ASD - N.B. all you need is in the two intro
videos - the data is available as Ass3\_MetaAnalysisData.tsv

\begin{Shaded}
\begin{Highlighting}[]
\CommentTok{\#We load the Meta Analysis Data that were provided into R}
\NormalTok{meta }\OtherTok{\textless{}{-}} \FunctionTok{read.table}\NormalTok{(}\AttributeTok{file =} \StringTok{\textquotesingle{}Ass3\_MetaAnalysisData.tsv\textquotesingle{}}\NormalTok{, }\AttributeTok{sep =} \StringTok{\textquotesingle{}}\SpecialCharTok{\textbackslash{}t}\StringTok{\textquotesingle{}}\NormalTok{, }\AttributeTok{header =} \ConstantTok{TRUE}\NormalTok{)}
\end{Highlighting}
\end{Shaded}

\begin{itemize}
\item
  You should calculate Effect size (cohen's d) and Standard Error
  (uncertainty in the Cohen's d) per each study, using escalc() from the
  metafor package (also check the livecoding intro)
\item
  N.B. for the purpose of the assignment we're only interested in
  getting a meta-analytic effect size for the meta-analytic prior (and
  not e.g.~all the stuff on publication bias). See a brms tutorial here:
  \url{https://vuorre.netlify.com/post/2016/09/29/meta-analysis-is-a-special-case-of-bayesian-multilevel-modeling/}
  The formula is EffectSize \textbar{} se(StandardError)
  \textasciitilde{} 1 + (1 \textbar{} Paper). Don't forget prior
  definition, model checking, etc.
\item
  N.B. the livecoding video is not perfect, you \emph{can} (but don't
  have to) improve it: use a t-student likelihood, test the effects of
  the priors and refine them, check the robustness of results to more or
  less skeptical priors, etc.
\item
  Write down the results of the meta-analysis in terms of a prior for
  step 2.
\end{itemize}

\begin{Shaded}
\begin{Highlighting}[]
\CommentTok{\#Morten}

\CommentTok{\#We load the Meta Analysis Data that were provided into R}
\NormalTok{meta }\OtherTok{\textless{}{-}} \FunctionTok{read.table}\NormalTok{(}\AttributeTok{file =} \StringTok{\textquotesingle{}Ass3\_MetaAnalysisData.tsv\textquotesingle{}}\NormalTok{, }\AttributeTok{sep =} \StringTok{\textquotesingle{}}\SpecialCharTok{\textbackslash{}t}\StringTok{\textquotesingle{}}\NormalTok{, }\AttributeTok{header =} \ConstantTok{TRUE}\NormalTok{)}

\CommentTok{\# Make variables numeric}
\NormalTok{meta }\OtherTok{\textless{}{-}}\NormalTok{ meta }\SpecialCharTok{\%\textgreater{}\%} 
  \FunctionTok{mutate}\NormalTok{(}
    \AttributeTok{PitchVariabilityASD\_Mean =} \FunctionTok{as.numeric}\NormalTok{(PitchVariabilityASD\_Mean),}
    \AttributeTok{PitchVariabilityTD\_Mean =} \FunctionTok{as.numeric}\NormalTok{(PitchVariabilityTD\_Mean),}
    \AttributeTok{PitchVariabilityASD\_SD =} \FunctionTok{as.numeric}\NormalTok{(PitchVariabilityASD\_SD),}
    \AttributeTok{PitchVariabilityTD\_SD =} \FunctionTok{as.numeric}\NormalTok{(PitchVariabilityTD\_SD)}
\NormalTok{  )}

\CommentTok{\# Remove rows with NA in paper}
\NormalTok{meta }\OtherTok{\textless{}{-}}\NormalTok{ meta }\SpecialCharTok{\%\textgreater{}\%} \FunctionTok{subset}\NormalTok{(}\SpecialCharTok{!}\FunctionTok{is.na}\NormalTok{(Paper))}

\CommentTok{\# Calculating effect size and standard error per study, SMD = Standardized Mean Difference, n1i and n2i are group sizes, m1i and m2i are means of groups, sd1i and sd2i are standard deviation of groups}
\NormalTok{meta }\OtherTok{\textless{}{-}} \FunctionTok{escalc}\NormalTok{(}\AttributeTok{measure =} \StringTok{\textquotesingle{}SMD\textquotesingle{}}\NormalTok{, }
                      \AttributeTok{n1i =}\NormalTok{ TD\_N, }
                      \AttributeTok{n2i =}\NormalTok{ ASD\_N,}
                      \AttributeTok{m1i =}\NormalTok{ PitchVariabilityTD\_Mean, }
                      \AttributeTok{m2i =}\NormalTok{ PitchVariabilityASD\_Mean, }
                      \AttributeTok{sd1i =}\NormalTok{ PitchVariabilityTD\_SD, }
                      \AttributeTok{sd2i =}\NormalTok{ PitchVariabilityASD\_SD, }
                      \AttributeTok{data =}\NormalTok{ meta,}
                      \AttributeTok{slab =}\NormalTok{ Paper)}

\CommentTok{\# renaming to effect size and standard error}
\NormalTok{meta }\OtherTok{\textless{}{-}}\NormalTok{ meta }\SpecialCharTok{\%\textgreater{}\%} \FunctionTok{mutate}\NormalTok{(}
  \AttributeTok{StandardError =} \FunctionTok{sqrt}\NormalTok{(vi)}
\NormalTok{  ) }\SpecialCharTok{\%\textgreater{}\%} \FunctionTok{rename}\NormalTok{(}
  \AttributeTok{EffectSize =}\NormalTok{ yi}
\NormalTok{)}

\CommentTok{\#Nicoline}
\CommentTok{\# checking effect size and standard error}
\FunctionTok{summary}\NormalTok{(meta}\SpecialCharTok{$}\NormalTok{EffectSize)}
\end{Highlighting}
\end{Shaded}

\begin{verbatim}
##     Min.  1st Qu.   Median     Mean  3rd Qu.     Max.     NA's 
## -1.29110 -0.81658 -0.65338 -0.46315 -0.05907  0.52031       17
\end{verbatim}

\begin{Shaded}
\begin{Highlighting}[]
\FunctionTok{summary}\NormalTok{(meta}\SpecialCharTok{$}\NormalTok{StandardError)}
\end{Highlighting}
\end{Shaded}

\begin{verbatim}
##    Min. 1st Qu.  Median    Mean 3rd Qu.    Max.    NA's 
##  0.2211  0.3176  0.3732  0.3673  0.4243  0.4826      17
\end{verbatim}

\begin{Shaded}
\begin{Highlighting}[]
\CommentTok{\# creating formula for model for effect size with population as random intercept}
\NormalTok{meta\_f }\OtherTok{\textless{}{-}} \FunctionTok{bf}\NormalTok{(EffectSize }\SpecialCharTok{|} \FunctionTok{se}\NormalTok{(StandardError) }\SpecialCharTok{\textasciitilde{}} \DecValTok{1} \SpecialCharTok{+}\NormalTok{ (}\DecValTok{1} \SpecialCharTok{|}\NormalTok{ Population))}

\CommentTok{\# get priors for model}
\FunctionTok{get\_prior}\NormalTok{(meta\_f, }\AttributeTok{data =}\NormalTok{ meta, }\AttributeTok{family =} \FunctionTok{gaussian}\NormalTok{())}
\end{Highlighting}
\end{Shaded}

\begin{verbatim}
## Warning: Rows containing NAs were excluded from the model.
\end{verbatim}

\begin{verbatim}
##                    prior     class      coef      group resp dpar nlpar bound
##  student_t(3, -0.7, 2.5) Intercept                                           
##     student_t(3, 0, 2.5)        sd                                           
##     student_t(3, 0, 2.5)        sd           Population                      
##     student_t(3, 0, 2.5)        sd Intercept Population                      
##        source
##       default
##       default
##  (vectorized)
##  (vectorized)
\end{verbatim}

\begin{Shaded}
\begin{Highlighting}[]
\CommentTok{\# make priors}
\NormalTok{meta\_prior }\OtherTok{\textless{}{-}} \FunctionTok{c}\NormalTok{(}
  \FunctionTok{prior}\NormalTok{(}\FunctionTok{normal}\NormalTok{(}\DecValTok{0}\NormalTok{, }\DecValTok{1}\NormalTok{), }\AttributeTok{class =}\NormalTok{ Intercept),}
  \FunctionTok{prior}\NormalTok{(}\FunctionTok{normal}\NormalTok{(}\DecValTok{0}\NormalTok{, }\FloatTok{0.3}\NormalTok{), }\AttributeTok{class =}\NormalTok{ sd)}
\NormalTok{)}

\CommentTok{\# run model with above formula and priors without using data}
\NormalTok{meta\_m0 }\OtherTok{\textless{}{-}} \FunctionTok{brm}\NormalTok{(}
\NormalTok{  meta\_f,}
  \AttributeTok{data =}\NormalTok{ meta,}
  \AttributeTok{family =} \FunctionTok{gaussian}\NormalTok{(),}
  \AttributeTok{prior =}\NormalTok{ meta\_prior,}
  \AttributeTok{sample\_prior =} \StringTok{"only"}\NormalTok{,}
  \AttributeTok{chains =} \DecValTok{2}\NormalTok{,}
  \AttributeTok{cores =} \DecValTok{2}\NormalTok{,}
  \AttributeTok{backend =} \StringTok{"cmdstanr"}\NormalTok{,}
  \AttributeTok{threads =} \FunctionTok{threading}\NormalTok{(}\DecValTok{2}\NormalTok{)}
\NormalTok{)}
\end{Highlighting}
\end{Shaded}

\begin{verbatim}
## Warning: Rows containing NAs were excluded from the model.
\end{verbatim}

\begin{verbatim}
## Compiling Stan program...
\end{verbatim}

\begin{verbatim}
## In file included from stan/lib/stan_math/stan/math/rev/functor.hpp:28,
\end{verbatim}

\begin{verbatim}
##                  from stan/lib/stan_math/stan/math/rev.hpp:11,
\end{verbatim}

\begin{verbatim}
##                  from stan/lib/stan_math/stan/math.hpp:19,
\end{verbatim}

\begin{verbatim}
##                  from stan/src/stan/model/model_header.hpp:4,
\end{verbatim}

\begin{verbatim}
##                  from C:/Users/ukoto/AppData/Local/Temp/RtmpCcFj0B/model-206c28cd5254.hpp:3:
\end{verbatim}

\begin{verbatim}
## stan/lib/stan_math/stan/math/rev/functor/reduce_sum.hpp: In instantiation of 'stan::math::internal::reduce_sum_impl<ReduceFunction, typename std::enable_if<stan::is_var<typename std::decay<_Arg>::type, void>::value, void>::type, ReturnType, Vec, Args ...>::recursive_reducer::recursive_reducer(size_t, size_t, double*, VecT&&, ArgsT&& ...) [with VecT = const std::vector<int>&; ArgsT = {const Eigen::Matrix<double, -1, 1, 0, -1, 1>&, const Eigen::Matrix<double, -1, 1, 0, -1, 1>&, const Eigen::Matrix<double, -1, 1, 0, -1, 1>&, stan::math::var_value<double, void>&, stan::math::var_value<double, void>&, const std::vector<int, std::allocator<int> >&, const Eigen::Matrix<double, -1, 1, 0, -1, 1>&, Eigen::Matrix<stan::math::var_value<double, void>, -1, 1, 0, -1, 1>&}; ReduceFunction = file206c167e1c31_model_namespace::partial_log_lik_rsfunctor__; ReturnType = stan::math::var_value<double>; Vec = const std::vector<int>&; Args = {const Eigen::Matrix<double, -1, 1, 0, -1, 1>&, const Eigen::Matrix<double, -1, 1, 0, -1, 1>&, const Eigen::Matrix<double, -1, 1, 0, -1, 1>&, stan::math::var_value<double, void>&, stan::math::var_value<double, void>&, const std::vector<int, std::allocator<int> >&, const Eigen::Matrix<double, -1, 1, 0, -1, 1>&, Eigen::Matrix<stan::math::var_value<double, void>, -1, 1, 0, -1, 1>&}; size_t = long long unsigned int]':
\end{verbatim}

\begin{verbatim}
## stan/lib/stan_math/stan/math/rev/functor/reduce_sum.hpp:248:23:   required from 'stan::math::var stan::math::internal::reduce_sum_impl<ReduceFunction, typename std::enable_if<stan::is_var<typename std::decay<_Arg>::type, void>::value, void>::type, ReturnType, Vec, Args ...>::operator()(Vec&&, bool, int, std::ostream*, Args&& ...) const [with ReduceFunction = file206c167e1c31_model_namespace::partial_log_lik_rsfunctor__; ReturnType = stan::math::var_value<double>; Vec = const std::vector<int>&; Args = {const Eigen::Matrix<double, -1, 1, 0, -1, 1>&, const Eigen::Matrix<double, -1, 1, 0, -1, 1>&, const Eigen::Matrix<double, -1, 1, 0, -1, 1>&, stan::math::var_value<double, void>&, stan::math::var_value<double, void>&, const std::vector<int, std::allocator<int> >&, const Eigen::Matrix<double, -1, 1, 0, -1, 1>&, Eigen::Matrix<stan::math::var_value<double, void>, -1, 1, 0, -1, 1>&}; typename std::enable_if<stan::is_var<typename std::decay<_Arg>::type, void>::value, void>::type = void; stan::math::var = stan::math::var_value<double>; std::ostream = std::basic_ostream<char>]'
\end{verbatim}

\begin{verbatim}
## stan/lib/stan_math/stan/math/prim/functor/reduce_sum.hpp:207:46:   required from 'auto stan::math::reduce_sum(Vec&&, int, std::ostream*, Args&& ...) [with ReduceFunction = file206c167e1c31_model_namespace::partial_log_lik_rsfunctor__; Vec = const std::vector<int>&; <template-parameter-1-3> = void; Args = {const Eigen::Matrix<double, -1, 1, 0, -1, 1>&, const Eigen::Matrix<double, -1, 1, 0, -1, 1>&, const Eigen::Matrix<double, -1, 1, 0, -1, 1>&, stan::math::var_value<double, void>&, stan::math::var_value<double, void>&, const std::vector<int, std::allocator<int> >&, const Eigen::Matrix<double, -1, 1, 0, -1, 1>&, Eigen::Matrix<stan::math::var_value<double, void>, -1, 1, 0, -1, 1>&}; std::ostream = std::basic_ostream<char>]'
\end{verbatim}

\begin{verbatim}
## C:/Users/ukoto/AppData/Local/Temp/RtmpCcFj0B/model-206c28cd5254.hpp:562:52:   required from 'stan::scalar_type_t<T2> file206c167e1c31_model_namespace::file206c167e1c31_model::log_prob_impl(VecR&, VecI&, std::ostream*) const [with bool propto__ = false; bool jacobian__ = false; VecR = Eigen::Matrix<stan::math::var_value<double>, -1, 1>; VecI = Eigen::Matrix<int, -1, 1>; stan::require_vector_like_t<VecR>* <anonymous> = 0; stan::require_vector_like_vt<std::is_integral, VecI>* <anonymous> = 0; stan::scalar_type_t<T2> = stan::math::var_value<double>; std::ostream = std::basic_ostream<char>]'
\end{verbatim}

\begin{verbatim}
## C:/Users/ukoto/AppData/Local/Temp/RtmpCcFj0B/model-206c28cd5254.hpp:896:77:   required from 'T_ file206c167e1c31_model_namespace::file206c167e1c31_model::log_prob(Eigen::Matrix<T_job_param, -1, 1>&, std::ostream*) const [with bool propto__ = false; bool jacobian__ = false; T_ = stan::math::var_value<double>; std::ostream = std::basic_ostream<char>]'
\end{verbatim}

\begin{verbatim}
## stan/src/stan/model/model_base_crtp.hpp:96:77:   required from 'stan::math::var stan::model::model_base_crtp<M>::log_prob(Eigen::Matrix<stan::math::var_value<double>, -1, 1>&, std::ostream*) const [with M = file206c167e1c31_model_namespace::file206c167e1c31_model; stan::math::var = stan::math::var_value<double>; std::ostream = std::basic_ostream<char>]'
\end{verbatim}

\begin{verbatim}
## stan/src/stan/model/model_base_crtp.hpp:93:20:   required from here
\end{verbatim}

\begin{verbatim}
## stan/lib/stan_math/stan/math/rev/functor/reduce_sum.hpp:57:23: warning: 'stan::math::internal::reduce_sum_impl<file206c167e1c31_model_namespace::partial_log_lik_rsfunctor__, void, stan::math::var_value<double>, const std::vector<int>&, const Eigen::Matrix<double, -1, 1, 0, -1, 1>&, const Eigen::Matrix<double, -1, 1, 0, -1, 1>&, const Eigen::Matrix<double, -1, 1, 0, -1, 1>&, stan::math::var_value<double, void>&, stan::math::var_value<double, void>&, const std::vector<int, std::allocator<int> >&, const Eigen::Matrix<double, -1, 1, 0, -1, 1>&, Eigen::Matrix<stan::math::var_value<double, void>, -1, 1, 0, -1, 1>&>::recursive_reducer::local_args_tuple_scope_' will be initialized after [-Wreorder]
\end{verbatim}

\begin{verbatim}
##      scoped_args_tuple local_args_tuple_scope_;
\end{verbatim}

\begin{verbatim}
##                        ^~~~~~~~~~~~~~~~~~~~~~~
\end{verbatim}

\begin{verbatim}
## stan/lib/stan_math/stan/math/rev/functor/reduce_sum.hpp:56:25: warning:   'std::tuple<const Eigen::Matrix<double, -1, 1, 0, -1, 1>&, const Eigen::Matrix<double, -1, 1, 0, -1, 1>&, const Eigen::Matrix<double, -1, 1, 0, -1, 1>&, stan::math::var_value<double, void>&, stan::math::var_value<double, void>&, const std::vector<int, std::allocator<int> >&, const Eigen::Matrix<double, -1, 1, 0, -1, 1>&, Eigen::Matrix<stan::math::var_value<double, void>, -1, 1, 0, -1, 1>&> stan::math::internal::reduce_sum_impl<file206c167e1c31_model_namespace::partial_log_lik_rsfunctor__, void, stan::math::var_value<double>, const std::vector<int>&, const Eigen::Matrix<double, -1, 1, 0, -1, 1>&, const Eigen::Matrix<double, -1, 1, 0, -1, 1>&, const Eigen::Matrix<double, -1, 1, 0, -1, 1>&, stan::math::var_value<double, void>&, stan::math::var_value<double, void>&, const std::vector<int, std::allocator<int> >&, const Eigen::Matrix<double, -1, 1, 0, -1, 1>&, Eigen::Matrix<stan::math::var_value<double, void>, -1, 1, 0, -1, 1>&>::recursive_reducer::args_tuple_' [-Wreorder]
\end{verbatim}

\begin{verbatim}
##      std::tuple<Args...> args_tuple_;
\end{verbatim}

\begin{verbatim}
##                          ^~~~~~~~~~~
\end{verbatim}

\begin{verbatim}
## stan/lib/stan_math/stan/math/rev/functor/reduce_sum.hpp:62:5: warning:   when initialized here [-Wreorder]
\end{verbatim}

\begin{verbatim}
##      recursive_reducer(size_t num_vars_per_term, size_t num_vars_shared_terms,
\end{verbatim}

\begin{verbatim}
##      ^~~~~~~~~~~~~~~~~
\end{verbatim}

\begin{verbatim}
## stan/lib/stan_math/stan/math/rev/functor/reduce_sum.hpp: In instantiation of 'stan::math::internal::reduce_sum_impl<ReduceFunction, typename std::enable_if<stan::is_var<typename std::decay<_Arg>::type, void>::value, void>::type, ReturnType, Vec, Args ...>::recursive_reducer::recursive_reducer(stan::math::internal::reduce_sum_impl<ReduceFunction, typename std::enable_if<stan::is_var<typename std::decay<_Arg>::type, void>::value, void>::type, ReturnType, Vec, Args ...>::recursive_reducer&, tbb::split) [with ReduceFunction = file206c167e1c31_model_namespace::partial_log_lik_rsfunctor__; ReturnType = stan::math::var_value<double>; Vec = const std::vector<int>&; Args = {const Eigen::Matrix<double, -1, 1, 0, -1, 1>&, const Eigen::Matrix<double, -1, 1, 0, -1, 1>&, const Eigen::Matrix<double, -1, 1, 0, -1, 1>&, stan::math::var_value<double, void>&, stan::math::var_value<double, void>&, const std::vector<int, std::allocator<int> >&, const Eigen::Matrix<double, -1, 1, 0, -1, 1>&, Eigen::Matrix<stan::math::var_value<double, void>, -1, 1, 0, -1, 1>&}]':
\end{verbatim}

\begin{verbatim}
## stan/lib/stan_math/lib/tbb_2019_U8/include/tbb/parallel_reduce.h:183:27:   required from 'tbb::task* tbb::interface9::internal::start_reduce<Range, Body, Partitioner>::execute() [with Range = tbb::blocked_range<long long unsigned int>; Body = stan::math::internal::reduce_sum_impl<file206c167e1c31_model_namespace::partial_log_lik_rsfunctor__, void, stan::math::var_value<double>, const std::vector<int>&, const Eigen::Matrix<double, -1, 1, 0, -1, 1>&, const Eigen::Matrix<double, -1, 1, 0, -1, 1>&, const Eigen::Matrix<double, -1, 1, 0, -1, 1>&, stan::math::var_value<double, void>&, stan::math::var_value<double, void>&, const std::vector<int, std::allocator<int> >&, const Eigen::Matrix<double, -1, 1, 0, -1, 1>&, Eigen::Matrix<stan::math::var_value<double, void>, -1, 1, 0, -1, 1>&>::recursive_reducer; Partitioner = const tbb::auto_partitioner]'
\end{verbatim}

\begin{verbatim}
## stan/lib/stan_math/lib/tbb_2019_U8/include/tbb/parallel_reduce.h:178:11:   required from here
\end{verbatim}

\begin{verbatim}
## stan/lib/stan_math/stan/math/rev/functor/reduce_sum.hpp:57:23: warning: 'stan::math::internal::reduce_sum_impl<file206c167e1c31_model_namespace::partial_log_lik_rsfunctor__, void, stan::math::var_value<double>, const std::vector<int>&, const Eigen::Matrix<double, -1, 1, 0, -1, 1>&, const Eigen::Matrix<double, -1, 1, 0, -1, 1>&, const Eigen::Matrix<double, -1, 1, 0, -1, 1>&, stan::math::var_value<double, void>&, stan::math::var_value<double, void>&, const std::vector<int, std::allocator<int> >&, const Eigen::Matrix<double, -1, 1, 0, -1, 1>&, Eigen::Matrix<stan::math::var_value<double, void>, -1, 1, 0, -1, 1>&>::recursive_reducer::local_args_tuple_scope_' will be initialized after [-Wreorder]
\end{verbatim}

\begin{verbatim}
##      scoped_args_tuple local_args_tuple_scope_;
\end{verbatim}

\begin{verbatim}
##                        ^~~~~~~~~~~~~~~~~~~~~~~
\end{verbatim}

\begin{verbatim}
## stan/lib/stan_math/stan/math/rev/functor/reduce_sum.hpp:56:25: warning:   'std::tuple<const Eigen::Matrix<double, -1, 1, 0, -1, 1>&, const Eigen::Matrix<double, -1, 1, 0, -1, 1>&, const Eigen::Matrix<double, -1, 1, 0, -1, 1>&, stan::math::var_value<double, void>&, stan::math::var_value<double, void>&, const std::vector<int, std::allocator<int> >&, const Eigen::Matrix<double, -1, 1, 0, -1, 1>&, Eigen::Matrix<stan::math::var_value<double, void>, -1, 1, 0, -1, 1>&> stan::math::internal::reduce_sum_impl<file206c167e1c31_model_namespace::partial_log_lik_rsfunctor__, void, stan::math::var_value<double>, const std::vector<int>&, const Eigen::Matrix<double, -1, 1, 0, -1, 1>&, const Eigen::Matrix<double, -1, 1, 0, -1, 1>&, const Eigen::Matrix<double, -1, 1, 0, -1, 1>&, stan::math::var_value<double, void>&, stan::math::var_value<double, void>&, const std::vector<int, std::allocator<int> >&, const Eigen::Matrix<double, -1, 1, 0, -1, 1>&, Eigen::Matrix<stan::math::var_value<double, void>, -1, 1, 0, -1, 1>&>::recursive_reducer::args_tuple_' [-Wreorder]
\end{verbatim}

\begin{verbatim}
##      std::tuple<Args...> args_tuple_;
\end{verbatim}

\begin{verbatim}
##                          ^~~~~~~~~~~
\end{verbatim}

\begin{verbatim}
## stan/lib/stan_math/stan/math/rev/functor/reduce_sum.hpp:77:5: warning:   when initialized here [-Wreorder]
\end{verbatim}

\begin{verbatim}
##      recursive_reducer(recursive_reducer& other, tbb::split)
\end{verbatim}

\begin{verbatim}
##      ^~~~~~~~~~~~~~~~~
\end{verbatim}

\begin{verbatim}
## Start sampling
\end{verbatim}

\begin{verbatim}
## Running MCMC with 2 parallel chains, with 2 thread(s) per chain...
## 
## Chain 1 Iteration:    1 / 2000 [  0%]  (Warmup) 
## Chain 1 Iteration:  100 / 2000 [  5%]  (Warmup) 
## Chain 1 Iteration:  200 / 2000 [ 10%]  (Warmup) 
## Chain 1 Iteration:  300 / 2000 [ 15%]  (Warmup) 
## Chain 1 Iteration:  400 / 2000 [ 20%]  (Warmup) 
## Chain 1 Iteration:  500 / 2000 [ 25%]  (Warmup) 
## Chain 1 Iteration:  600 / 2000 [ 30%]  (Warmup) 
## Chain 1 Iteration:  700 / 2000 [ 35%]  (Warmup) 
## Chain 1 Iteration:  800 / 2000 [ 40%]  (Warmup) 
## Chain 1 Iteration:  900 / 2000 [ 45%]  (Warmup) 
## Chain 1 Iteration: 1000 / 2000 [ 50%]  (Warmup) 
## Chain 1 Iteration: 1001 / 2000 [ 50%]  (Sampling) 
## Chain 1 Iteration: 1100 / 2000 [ 55%]  (Sampling) 
## Chain 2 Iteration:    1 / 2000 [  0%]  (Warmup) 
## Chain 2 Iteration:  100 / 2000 [  5%]  (Warmup) 
## Chain 2 Iteration:  200 / 2000 [ 10%]  (Warmup) 
## Chain 2 Iteration:  300 / 2000 [ 15%]  (Warmup) 
## Chain 2 Iteration:  400 / 2000 [ 20%]  (Warmup) 
## Chain 2 Iteration:  500 / 2000 [ 25%]  (Warmup) 
## Chain 2 Iteration:  600 / 2000 [ 30%]  (Warmup) 
## Chain 2 Iteration:  700 / 2000 [ 35%]  (Warmup) 
## Chain 2 Iteration:  800 / 2000 [ 40%]  (Warmup) 
## Chain 2 Iteration:  900 / 2000 [ 45%]  (Warmup) 
## Chain 2 Iteration: 1000 / 2000 [ 50%]  (Warmup) 
## Chain 2 Iteration: 1001 / 2000 [ 50%]  (Sampling) 
## Chain 1 Iteration: 1200 / 2000 [ 60%]  (Sampling) 
## Chain 1 Iteration: 1300 / 2000 [ 65%]  (Sampling) 
## Chain 1 Iteration: 1400 / 2000 [ 70%]  (Sampling) 
## Chain 1 Iteration: 1500 / 2000 [ 75%]  (Sampling) 
## Chain 1 Iteration: 1600 / 2000 [ 80%]  (Sampling) 
## Chain 1 Iteration: 1700 / 2000 [ 85%]  (Sampling) 
## Chain 1 Iteration: 1800 / 2000 [ 90%]  (Sampling) 
## Chain 1 Iteration: 1900 / 2000 [ 95%]  (Sampling) 
## Chain 1 Iteration: 2000 / 2000 [100%]  (Sampling) 
## Chain 2 Iteration: 1100 / 2000 [ 55%]  (Sampling) 
## Chain 2 Iteration: 1200 / 2000 [ 60%]  (Sampling) 
## Chain 2 Iteration: 1300 / 2000 [ 65%]  (Sampling) 
## Chain 2 Iteration: 1400 / 2000 [ 70%]  (Sampling) 
## Chain 2 Iteration: 1500 / 2000 [ 75%]  (Sampling) 
## Chain 2 Iteration: 1600 / 2000 [ 80%]  (Sampling) 
## Chain 2 Iteration: 1700 / 2000 [ 85%]  (Sampling) 
## Chain 2 Iteration: 1800 / 2000 [ 90%]  (Sampling) 
## Chain 2 Iteration: 1900 / 2000 [ 95%]  (Sampling) 
## Chain 2 Iteration: 2000 / 2000 [100%]  (Sampling) 
## Chain 1 finished in 0.2 seconds.
## Chain 2 finished in 0.2 seconds.
## 
## Both chains finished successfully.
## Mean chain execution time: 0.2 seconds.
## Total execution time: 0.4 seconds.
\end{verbatim}

\begin{Shaded}
\begin{Highlighting}[]
\CommentTok{\# prior predictive check}
\FunctionTok{pp\_check}\NormalTok{(meta\_m0, }\AttributeTok{nsamples =} \DecValTok{100}\NormalTok{)}
\end{Highlighting}
\end{Shaded}

\includegraphics{Assignment3-final_files/figure-latex/unnamed-chunk-2-1.pdf}

\begin{Shaded}
\begin{Highlighting}[]
\CommentTok{\# running same model, this time using data}
\NormalTok{meta\_m1 }\OtherTok{\textless{}{-}} \FunctionTok{brm}\NormalTok{(}
\NormalTok{  meta\_f,}
  \AttributeTok{data =}\NormalTok{ meta,}
  \AttributeTok{family =} \FunctionTok{gaussian}\NormalTok{(),}
  \AttributeTok{prior =}\NormalTok{ meta\_prior,}
  \AttributeTok{sample\_prior =}\NormalTok{ T,}
  \AttributeTok{chains =} \DecValTok{2}\NormalTok{,}
  \AttributeTok{cores =} \DecValTok{2}\NormalTok{, }
  \AttributeTok{backend =} \StringTok{"cmdstanr"}\NormalTok{,}
  \AttributeTok{threads =} \FunctionTok{threading}\NormalTok{(}\DecValTok{2}\NormalTok{)}
\NormalTok{)}
\end{Highlighting}
\end{Shaded}

\begin{verbatim}
## Warning: Rows containing NAs were excluded from the model.
\end{verbatim}

\begin{verbatim}
## Compiling Stan program...
\end{verbatim}

\begin{verbatim}
## In file included from stan/lib/stan_math/stan/math/rev/functor.hpp:28,
\end{verbatim}

\begin{verbatim}
##                  from stan/lib/stan_math/stan/math/rev.hpp:11,
\end{verbatim}

\begin{verbatim}
##                  from stan/lib/stan_math/stan/math.hpp:19,
\end{verbatim}

\begin{verbatim}
##                  from stan/src/stan/model/model_header.hpp:4,
\end{verbatim}

\begin{verbatim}
##                  from C:/Users/ukoto/AppData/Local/Temp/RtmpCcFj0B/model-206c5e0d2d4c.hpp:3:
\end{verbatim}

\begin{verbatim}
## stan/lib/stan_math/stan/math/rev/functor/reduce_sum.hpp: In instantiation of 'stan::math::internal::reduce_sum_impl<ReduceFunction, typename std::enable_if<stan::is_var<typename std::decay<_Arg>::type, void>::value, void>::type, ReturnType, Vec, Args ...>::recursive_reducer::recursive_reducer(size_t, size_t, double*, VecT&&, ArgsT&& ...) [with VecT = const std::vector<int>&; ArgsT = {const Eigen::Matrix<double, -1, 1, 0, -1, 1>&, const Eigen::Matrix<double, -1, 1, 0, -1, 1>&, const Eigen::Matrix<double, -1, 1, 0, -1, 1>&, stan::math::var_value<double, void>&, stan::math::var_value<double, void>&, const std::vector<int, std::allocator<int> >&, const Eigen::Matrix<double, -1, 1, 0, -1, 1>&, Eigen::Matrix<stan::math::var_value<double, void>, -1, 1, 0, -1, 1>&}; ReduceFunction = file206c1eb27870_model_namespace::partial_log_lik_rsfunctor__; ReturnType = stan::math::var_value<double>; Vec = const std::vector<int>&; Args = {const Eigen::Matrix<double, -1, 1, 0, -1, 1>&, const Eigen::Matrix<double, -1, 1, 0, -1, 1>&, const Eigen::Matrix<double, -1, 1, 0, -1, 1>&, stan::math::var_value<double, void>&, stan::math::var_value<double, void>&, const std::vector<int, std::allocator<int> >&, const Eigen::Matrix<double, -1, 1, 0, -1, 1>&, Eigen::Matrix<stan::math::var_value<double, void>, -1, 1, 0, -1, 1>&}; size_t = long long unsigned int]':
\end{verbatim}

\begin{verbatim}
## stan/lib/stan_math/stan/math/rev/functor/reduce_sum.hpp:248:23:   required from 'stan::math::var stan::math::internal::reduce_sum_impl<ReduceFunction, typename std::enable_if<stan::is_var<typename std::decay<_Arg>::type, void>::value, void>::type, ReturnType, Vec, Args ...>::operator()(Vec&&, bool, int, std::ostream*, Args&& ...) const [with ReduceFunction = file206c1eb27870_model_namespace::partial_log_lik_rsfunctor__; ReturnType = stan::math::var_value<double>; Vec = const std::vector<int>&; Args = {const Eigen::Matrix<double, -1, 1, 0, -1, 1>&, const Eigen::Matrix<double, -1, 1, 0, -1, 1>&, const Eigen::Matrix<double, -1, 1, 0, -1, 1>&, stan::math::var_value<double, void>&, stan::math::var_value<double, void>&, const std::vector<int, std::allocator<int> >&, const Eigen::Matrix<double, -1, 1, 0, -1, 1>&, Eigen::Matrix<stan::math::var_value<double, void>, -1, 1, 0, -1, 1>&}; typename std::enable_if<stan::is_var<typename std::decay<_Arg>::type, void>::value, void>::type = void; stan::math::var = stan::math::var_value<double>; std::ostream = std::basic_ostream<char>]'
\end{verbatim}

\begin{verbatim}
## stan/lib/stan_math/stan/math/prim/functor/reduce_sum.hpp:207:46:   required from 'auto stan::math::reduce_sum(Vec&&, int, std::ostream*, Args&& ...) [with ReduceFunction = file206c1eb27870_model_namespace::partial_log_lik_rsfunctor__; Vec = const std::vector<int>&; <template-parameter-1-3> = void; Args = {const Eigen::Matrix<double, -1, 1, 0, -1, 1>&, const Eigen::Matrix<double, -1, 1, 0, -1, 1>&, const Eigen::Matrix<double, -1, 1, 0, -1, 1>&, stan::math::var_value<double, void>&, stan::math::var_value<double, void>&, const std::vector<int, std::allocator<int> >&, const Eigen::Matrix<double, -1, 1, 0, -1, 1>&, Eigen::Matrix<stan::math::var_value<double, void>, -1, 1, 0, -1, 1>&}; std::ostream = std::basic_ostream<char>]'
\end{verbatim}

\begin{verbatim}
## C:/Users/ukoto/AppData/Local/Temp/RtmpCcFj0B/model-206c5e0d2d4c.hpp:567:52:   required from 'stan::scalar_type_t<T2> file206c1eb27870_model_namespace::file206c1eb27870_model::log_prob_impl(VecR&, VecI&, std::ostream*) const [with bool propto__ = false; bool jacobian__ = false; VecR = Eigen::Matrix<stan::math::var_value<double>, -1, 1>; VecI = Eigen::Matrix<int, -1, 1>; stan::require_vector_like_t<VecR>* <anonymous> = 0; stan::require_vector_like_vt<std::is_integral, VecI>* <anonymous> = 0; stan::scalar_type_t<T2> = stan::math::var_value<double>; std::ostream = std::basic_ostream<char>]'
\end{verbatim}

\begin{verbatim}
## C:/Users/ukoto/AppData/Local/Temp/RtmpCcFj0B/model-206c5e0d2d4c.hpp:928:77:   required from 'T_ file206c1eb27870_model_namespace::file206c1eb27870_model::log_prob(Eigen::Matrix<T_job_param, -1, 1>&, std::ostream*) const [with bool propto__ = false; bool jacobian__ = false; T_ = stan::math::var_value<double>; std::ostream = std::basic_ostream<char>]'
\end{verbatim}

\begin{verbatim}
## stan/src/stan/model/model_base_crtp.hpp:96:77:   required from 'stan::math::var stan::model::model_base_crtp<M>::log_prob(Eigen::Matrix<stan::math::var_value<double>, -1, 1>&, std::ostream*) const [with M = file206c1eb27870_model_namespace::file206c1eb27870_model; stan::math::var = stan::math::var_value<double>; std::ostream = std::basic_ostream<char>]'
\end{verbatim}

\begin{verbatim}
## stan/src/stan/model/model_base_crtp.hpp:93:20:   required from here
\end{verbatim}

\begin{verbatim}
## stan/lib/stan_math/stan/math/rev/functor/reduce_sum.hpp:57:23: warning: 'stan::math::internal::reduce_sum_impl<file206c1eb27870_model_namespace::partial_log_lik_rsfunctor__, void, stan::math::var_value<double>, const std::vector<int>&, const Eigen::Matrix<double, -1, 1, 0, -1, 1>&, const Eigen::Matrix<double, -1, 1, 0, -1, 1>&, const Eigen::Matrix<double, -1, 1, 0, -1, 1>&, stan::math::var_value<double, void>&, stan::math::var_value<double, void>&, const std::vector<int, std::allocator<int> >&, const Eigen::Matrix<double, -1, 1, 0, -1, 1>&, Eigen::Matrix<stan::math::var_value<double, void>, -1, 1, 0, -1, 1>&>::recursive_reducer::local_args_tuple_scope_' will be initialized after [-Wreorder]
\end{verbatim}

\begin{verbatim}
##      scoped_args_tuple local_args_tuple_scope_;
\end{verbatim}

\begin{verbatim}
##                        ^~~~~~~~~~~~~~~~~~~~~~~
\end{verbatim}

\begin{verbatim}
## stan/lib/stan_math/stan/math/rev/functor/reduce_sum.hpp:56:25: warning:   'std::tuple<const Eigen::Matrix<double, -1, 1, 0, -1, 1>&, const Eigen::Matrix<double, -1, 1, 0, -1, 1>&, const Eigen::Matrix<double, -1, 1, 0, -1, 1>&, stan::math::var_value<double, void>&, stan::math::var_value<double, void>&, const std::vector<int, std::allocator<int> >&, const Eigen::Matrix<double, -1, 1, 0, -1, 1>&, Eigen::Matrix<stan::math::var_value<double, void>, -1, 1, 0, -1, 1>&> stan::math::internal::reduce_sum_impl<file206c1eb27870_model_namespace::partial_log_lik_rsfunctor__, void, stan::math::var_value<double>, const std::vector<int>&, const Eigen::Matrix<double, -1, 1, 0, -1, 1>&, const Eigen::Matrix<double, -1, 1, 0, -1, 1>&, const Eigen::Matrix<double, -1, 1, 0, -1, 1>&, stan::math::var_value<double, void>&, stan::math::var_value<double, void>&, const std::vector<int, std::allocator<int> >&, const Eigen::Matrix<double, -1, 1, 0, -1, 1>&, Eigen::Matrix<stan::math::var_value<double, void>, -1, 1, 0, -1, 1>&>::recursive_reducer::args_tuple_' [-Wreorder]
\end{verbatim}

\begin{verbatim}
##      std::tuple<Args...> args_tuple_;
\end{verbatim}

\begin{verbatim}
##                          ^~~~~~~~~~~
\end{verbatim}

\begin{verbatim}
## stan/lib/stan_math/stan/math/rev/functor/reduce_sum.hpp:62:5: warning:   when initialized here [-Wreorder]
\end{verbatim}

\begin{verbatim}
##      recursive_reducer(size_t num_vars_per_term, size_t num_vars_shared_terms,
\end{verbatim}

\begin{verbatim}
##      ^~~~~~~~~~~~~~~~~
\end{verbatim}

\begin{verbatim}
## stan/lib/stan_math/stan/math/rev/functor/reduce_sum.hpp: In instantiation of 'stan::math::internal::reduce_sum_impl<ReduceFunction, typename std::enable_if<stan::is_var<typename std::decay<_Arg>::type, void>::value, void>::type, ReturnType, Vec, Args ...>::recursive_reducer::recursive_reducer(stan::math::internal::reduce_sum_impl<ReduceFunction, typename std::enable_if<stan::is_var<typename std::decay<_Arg>::type, void>::value, void>::type, ReturnType, Vec, Args ...>::recursive_reducer&, tbb::split) [with ReduceFunction = file206c1eb27870_model_namespace::partial_log_lik_rsfunctor__; ReturnType = stan::math::var_value<double>; Vec = const std::vector<int>&; Args = {const Eigen::Matrix<double, -1, 1, 0, -1, 1>&, const Eigen::Matrix<double, -1, 1, 0, -1, 1>&, const Eigen::Matrix<double, -1, 1, 0, -1, 1>&, stan::math::var_value<double, void>&, stan::math::var_value<double, void>&, const std::vector<int, std::allocator<int> >&, const Eigen::Matrix<double, -1, 1, 0, -1, 1>&, Eigen::Matrix<stan::math::var_value<double, void>, -1, 1, 0, -1, 1>&}]':
\end{verbatim}

\begin{verbatim}
## stan/lib/stan_math/lib/tbb_2019_U8/include/tbb/parallel_reduce.h:183:27:   required from 'tbb::task* tbb::interface9::internal::start_reduce<Range, Body, Partitioner>::execute() [with Range = tbb::blocked_range<long long unsigned int>; Body = stan::math::internal::reduce_sum_impl<file206c1eb27870_model_namespace::partial_log_lik_rsfunctor__, void, stan::math::var_value<double>, const std::vector<int>&, const Eigen::Matrix<double, -1, 1, 0, -1, 1>&, const Eigen::Matrix<double, -1, 1, 0, -1, 1>&, const Eigen::Matrix<double, -1, 1, 0, -1, 1>&, stan::math::var_value<double, void>&, stan::math::var_value<double, void>&, const std::vector<int, std::allocator<int> >&, const Eigen::Matrix<double, -1, 1, 0, -1, 1>&, Eigen::Matrix<stan::math::var_value<double, void>, -1, 1, 0, -1, 1>&>::recursive_reducer; Partitioner = const tbb::auto_partitioner]'
\end{verbatim}

\begin{verbatim}
## stan/lib/stan_math/lib/tbb_2019_U8/include/tbb/parallel_reduce.h:178:11:   required from here
\end{verbatim}

\begin{verbatim}
## stan/lib/stan_math/stan/math/rev/functor/reduce_sum.hpp:57:23: warning: 'stan::math::internal::reduce_sum_impl<file206c1eb27870_model_namespace::partial_log_lik_rsfunctor__, void, stan::math::var_value<double>, const std::vector<int>&, const Eigen::Matrix<double, -1, 1, 0, -1, 1>&, const Eigen::Matrix<double, -1, 1, 0, -1, 1>&, const Eigen::Matrix<double, -1, 1, 0, -1, 1>&, stan::math::var_value<double, void>&, stan::math::var_value<double, void>&, const std::vector<int, std::allocator<int> >&, const Eigen::Matrix<double, -1, 1, 0, -1, 1>&, Eigen::Matrix<stan::math::var_value<double, void>, -1, 1, 0, -1, 1>&>::recursive_reducer::local_args_tuple_scope_' will be initialized after [-Wreorder]
\end{verbatim}

\begin{verbatim}
##      scoped_args_tuple local_args_tuple_scope_;
\end{verbatim}

\begin{verbatim}
##                        ^~~~~~~~~~~~~~~~~~~~~~~
\end{verbatim}

\begin{verbatim}
## stan/lib/stan_math/stan/math/rev/functor/reduce_sum.hpp:56:25: warning:   'std::tuple<const Eigen::Matrix<double, -1, 1, 0, -1, 1>&, const Eigen::Matrix<double, -1, 1, 0, -1, 1>&, const Eigen::Matrix<double, -1, 1, 0, -1, 1>&, stan::math::var_value<double, void>&, stan::math::var_value<double, void>&, const std::vector<int, std::allocator<int> >&, const Eigen::Matrix<double, -1, 1, 0, -1, 1>&, Eigen::Matrix<stan::math::var_value<double, void>, -1, 1, 0, -1, 1>&> stan::math::internal::reduce_sum_impl<file206c1eb27870_model_namespace::partial_log_lik_rsfunctor__, void, stan::math::var_value<double>, const std::vector<int>&, const Eigen::Matrix<double, -1, 1, 0, -1, 1>&, const Eigen::Matrix<double, -1, 1, 0, -1, 1>&, const Eigen::Matrix<double, -1, 1, 0, -1, 1>&, stan::math::var_value<double, void>&, stan::math::var_value<double, void>&, const std::vector<int, std::allocator<int> >&, const Eigen::Matrix<double, -1, 1, 0, -1, 1>&, Eigen::Matrix<stan::math::var_value<double, void>, -1, 1, 0, -1, 1>&>::recursive_reducer::args_tuple_' [-Wreorder]
\end{verbatim}

\begin{verbatim}
##      std::tuple<Args...> args_tuple_;
\end{verbatim}

\begin{verbatim}
##                          ^~~~~~~~~~~
\end{verbatim}

\begin{verbatim}
## stan/lib/stan_math/stan/math/rev/functor/reduce_sum.hpp:77:5: warning:   when initialized here [-Wreorder]
\end{verbatim}

\begin{verbatim}
##      recursive_reducer(recursive_reducer& other, tbb::split)
\end{verbatim}

\begin{verbatim}
##      ^~~~~~~~~~~~~~~~~
\end{verbatim}

\begin{verbatim}
## Start sampling
\end{verbatim}

\begin{verbatim}
## Running MCMC with 2 parallel chains, with 2 thread(s) per chain...
## 
## Chain 1 Iteration:    1 / 2000 [  0%]  (Warmup) 
## Chain 1 Iteration:  100 / 2000 [  5%]  (Warmup) 
## Chain 1 Iteration:  200 / 2000 [ 10%]  (Warmup) 
## Chain 1 Iteration:  300 / 2000 [ 15%]  (Warmup) 
## Chain 2 Iteration:    1 / 2000 [  0%]  (Warmup) 
## Chain 2 Iteration:  100 / 2000 [  5%]  (Warmup) 
## Chain 2 Iteration:  200 / 2000 [ 10%]  (Warmup) 
## Chain 1 Iteration:  400 / 2000 [ 20%]  (Warmup) 
## Chain 1 Iteration:  500 / 2000 [ 25%]  (Warmup) 
## Chain 1 Iteration:  600 / 2000 [ 30%]  (Warmup) 
## Chain 1 Iteration:  700 / 2000 [ 35%]  (Warmup) 
## Chain 1 Iteration:  800 / 2000 [ 40%]  (Warmup) 
## Chain 1 Iteration:  900 / 2000 [ 45%]  (Warmup) 
## Chain 1 Iteration: 1000 / 2000 [ 50%]  (Warmup) 
## Chain 1 Iteration: 1001 / 2000 [ 50%]  (Sampling) 
## Chain 1 Iteration: 1100 / 2000 [ 55%]  (Sampling) 
## Chain 1 Iteration: 1200 / 2000 [ 60%]  (Sampling) 
## Chain 1 Iteration: 1300 / 2000 [ 65%]  (Sampling) 
## Chain 2 Iteration:  300 / 2000 [ 15%]  (Warmup) 
## Chain 2 Iteration:  400 / 2000 [ 20%]  (Warmup) 
## Chain 2 Iteration:  500 / 2000 [ 25%]  (Warmup) 
## Chain 2 Iteration:  600 / 2000 [ 30%]  (Warmup) 
## Chain 2 Iteration:  700 / 2000 [ 35%]  (Warmup) 
## Chain 2 Iteration:  800 / 2000 [ 40%]  (Warmup) 
## Chain 2 Iteration:  900 / 2000 [ 45%]  (Warmup) 
## Chain 2 Iteration: 1000 / 2000 [ 50%]  (Warmup) 
## Chain 2 Iteration: 1001 / 2000 [ 50%]  (Sampling) 
## Chain 2 Iteration: 1100 / 2000 [ 55%]  (Sampling) 
## Chain 1 Iteration: 1400 / 2000 [ 70%]  (Sampling) 
## Chain 1 Iteration: 1500 / 2000 [ 75%]  (Sampling) 
## Chain 1 Iteration: 1600 / 2000 [ 80%]  (Sampling) 
## Chain 1 Iteration: 1700 / 2000 [ 85%]  (Sampling) 
## Chain 1 Iteration: 1800 / 2000 [ 90%]  (Sampling) 
## Chain 2 Iteration: 1200 / 2000 [ 60%]  (Sampling) 
## Chain 2 Iteration: 1300 / 2000 [ 65%]  (Sampling) 
## Chain 2 Iteration: 1400 / 2000 [ 70%]  (Sampling) 
## Chain 2 Iteration: 1500 / 2000 [ 75%]  (Sampling) 
## Chain 2 Iteration: 1600 / 2000 [ 80%]  (Sampling) 
## Chain 1 Iteration: 1900 / 2000 [ 95%]  (Sampling) 
## Chain 1 Iteration: 2000 / 2000 [100%]  (Sampling) 
## Chain 2 Iteration: 1700 / 2000 [ 85%]  (Sampling) 
## Chain 2 Iteration: 1800 / 2000 [ 90%]  (Sampling) 
## Chain 2 Iteration: 1900 / 2000 [ 95%]  (Sampling) 
## Chain 2 Iteration: 2000 / 2000 [100%]  (Sampling) 
## Chain 1 finished in 0.4 seconds.
## Chain 2 finished in 0.4 seconds.
## 
## Both chains finished successfully.
## Mean chain execution time: 0.4 seconds.
## Total execution time: 0.5 seconds.
\end{verbatim}

\begin{Shaded}
\begin{Highlighting}[]
\CommentTok{\# posterior predictive check}
\FunctionTok{pp\_check}\NormalTok{(meta\_m1, }\AttributeTok{nsamples =} \DecValTok{100}\NormalTok{)}
\end{Highlighting}
\end{Shaded}

\includegraphics{Assignment3-final_files/figure-latex/unnamed-chunk-2-2.pdf}

\begin{Shaded}
\begin{Highlighting}[]
\CommentTok{\# checking model}
\FunctionTok{summary}\NormalTok{(meta\_m1)}
\end{Highlighting}
\end{Shaded}

\begin{verbatim}
##  Family: gaussian 
##   Links: mu = identity; sigma = identity 
## Formula: EffectSize | se(StandardError) ~ 1 + (1 | Population) 
##    Data: meta (Number of observations: 30) 
## Samples: 2 chains, each with iter = 2000; warmup = 1000; thin = 1;
##          total post-warmup samples = 2000
## 
## Group-Level Effects: 
## ~Population (Number of levels: 26) 
##               Estimate Est.Error l-95% CI u-95% CI Rhat Bulk_ESS Tail_ESS
## sd(Intercept)     0.32      0.10     0.10     0.52 1.00      613      603
## 
## Population-Level Effects: 
##           Estimate Est.Error l-95% CI u-95% CI Rhat Bulk_ESS Tail_ESS
## Intercept    -0.44      0.10    -0.64    -0.26 1.00     1314      866
## 
## Family Specific Parameters: 
##       Estimate Est.Error l-95% CI u-95% CI Rhat Bulk_ESS Tail_ESS
## sigma     0.00      0.00     0.00     0.00 1.00     2000     2000
## 
## Samples were drawn using sample(hmc). For each parameter, Bulk_ESS
## and Tail_ESS are effective sample size measures, and Rhat is the potential
## scale reduction factor on split chains (at convergence, Rhat = 1).
\end{verbatim}

\begin{Shaded}
\begin{Highlighting}[]
\CommentTok{\# MA effect mean = 0.43, sd = 0.1}
\CommentTok{\# saving meta mean, meta standard error and meta effect size}
\NormalTok{meta\_mean }\OtherTok{\textless{}{-}} \FunctionTok{fixef}\NormalTok{(meta\_m1)[[}\DecValTok{1}\NormalTok{]]}
\NormalTok{meta\_error }\OtherTok{\textless{}{-}} \FunctionTok{fixef}\NormalTok{(meta\_m1)[[}\DecValTok{2}\NormalTok{]]}
\NormalTok{meta\_effectsize }\OtherTok{\textless{}{-}}\NormalTok{ .}\DecValTok{32}

\NormalTok{meta\_heterogeneity }\OtherTok{=} \FunctionTok{ranef}\NormalTok{(meta\_m1)}
\end{Highlighting}
\end{Shaded}

\#\#\#\#Step 2: Analyse pitch variability in ASD in two new studies for
which you have access to all the trials (not just study level estimates)
- the data is available as Ass3\_data.csv. Notice there are 2 studies
(language us, and language dk), multiple trials per participant, and a
few different ways to measure pitch variability (if in doubt, focus on
pitch IQR, interquartile range of the log of fundamental frequency)

\begin{Shaded}
\begin{Highlighting}[]
\CommentTok{\#We load the data from the two new studies into R}
\NormalTok{data }\OtherTok{\textless{}{-}} \FunctionTok{read\_csv}\NormalTok{(}\StringTok{"Ass3\_data.csv"}\NormalTok{)}
\end{Highlighting}
\end{Shaded}

\begin{verbatim}
## 
## -- Column specification --------------------------------------------------------
## cols(
##   .default = col_double(),
##   Language = col_character(),
##   Diagnosis = col_character(),
##   Gender = col_character(),
##   AdosCommunication = col_character(),
##   AdosSocial = col_character(),
##   AdosCreativity = col_character(),
##   AdosStereotyped = col_character(),
##   language = col_character()
## )
## i Use `spec()` for the full column specifications.
\end{verbatim}

\begin{verbatim}
## Warning: 32 parsing failures.
##  row col expected actual            file
## 1043  ID a double    ALY 'Ass3_data.csv'
## 1044  ID a double    ALY 'Ass3_data.csv'
## 1045  ID a double    ALY 'Ass3_data.csv'
## 1046  ID a double    ANS 'Ass3_data.csv'
## 1047  ID a double    ANS 'Ass3_data.csv'
## .... ... ........ ...... ...............
## See problems(...) for more details.
\end{verbatim}

\begin{itemize}
\tightlist
\item
  Also, let's standardize the data, so that they are compatible with our
  meta-analytic prior (Cohen's d is measured in SDs).
\end{itemize}

\begin{Shaded}
\begin{Highlighting}[]
\CommentTok{\#Magnus}
\CommentTok{\# Standardizing pitch IQE}
\NormalTok{data}\SpecialCharTok{$}\NormalTok{Pitch\_IQR\_stan }\OtherTok{\textless{}{-}}\NormalTok{ (data}\SpecialCharTok{$}\NormalTok{Pitch\_IQR }\SpecialCharTok{{-}} \FunctionTok{mean}\NormalTok{(data}\SpecialCharTok{$}\NormalTok{Pitch\_IQR))}\SpecialCharTok{/}\FunctionTok{sd}\NormalTok{(data}\SpecialCharTok{$}\NormalTok{Pitch\_IQR)}
\end{Highlighting}
\end{Shaded}

\begin{itemize}
\tightlist
\item
  Is there any structure in the dataset that we should account for with
  random/varying effects? How would you implement that? Or, if you don't
  know how to do bayesian random/varying effects or don't want to
  bother, is there anything we would need to simplify in the dataset?
\end{itemize}

\begin{Shaded}
\begin{Highlighting}[]
\CommentTok{\#Magnus}
\CommentTok{\# creating subset with only asd}
\NormalTok{asd }\OtherTok{\textless{}{-}}\NormalTok{ data }\SpecialCharTok{\%\textgreater{}\%} \FunctionTok{subset}\NormalTok{(Diagnosis }\SpecialCharTok{==} \StringTok{\textquotesingle{}ASD\textquotesingle{}}\NormalTok{)}

\CommentTok{\# creating subset with only td}
\NormalTok{td }\OtherTok{\textless{}{-}}\NormalTok{ data }\SpecialCharTok{\%\textgreater{}\%} \FunctionTok{subset}\NormalTok{(Diagnosis }\SpecialCharTok{==} \StringTok{\textquotesingle{}TD\textquotesingle{}}\NormalTok{)}

\CommentTok{\# checking mean duration for both groups}
\FunctionTok{mean}\NormalTok{(asd}\SpecialCharTok{$}\NormalTok{Duration)}
\end{Highlighting}
\end{Shaded}

\begin{verbatim}
## [1] 20.33454
\end{verbatim}

\begin{Shaded}
\begin{Highlighting}[]
\FunctionTok{mean}\NormalTok{(td}\SpecialCharTok{$}\NormalTok{Duration)}
\end{Highlighting}
\end{Shaded}

\begin{verbatim}
## [1] 17.20774
\end{verbatim}

\begin{Shaded}
\begin{Highlighting}[]
\CommentTok{\# checking mean speechrate for both groups}
\FunctionTok{mean}\NormalTok{(asd}\SpecialCharTok{$}\StringTok{\textasciigrave{}}\AttributeTok{speechrate (nsyll/dur)}\StringTok{\textasciigrave{}}\NormalTok{)}
\end{Highlighting}
\end{Shaded}

\begin{verbatim}
## [1] 3.048441
\end{verbatim}

\begin{Shaded}
\begin{Highlighting}[]
\FunctionTok{mean}\NormalTok{(td}\SpecialCharTok{$}\StringTok{\textasciigrave{}}\AttributeTok{speechrate (nsyll/dur)}\StringTok{\textasciigrave{}}\NormalTok{)}
\end{Highlighting}
\end{Shaded}

\begin{verbatim}
## [1] 3.267273
\end{verbatim}

\#\#\#\#Step 3: Build a regression model predicting Pitch variability
from Diagnosis. - how is the outcome distributed? (likelihood function).
NB. given we are standardizing, and the meta-analysis is on that scale,
gaussian is not a bad assumption, but check t-student as well. Lognormal
would require us to convert the prior to that scale. - how are the
parameters of the likelihood distribution distributed? Which predictors
should they be conditioned on? Start simple, with Diagnosis only. Add
other predictors only if you have the time and energy! - use a
skeptical/conservative prior for the effects of diagnosis. Remember
you'll need to motivate it, test its predictions (prior predictive
checks), its impact on the posteriors (prior posterior updates checks).
- Evaluate model quality. Describe and plot the estimates.

\begin{Shaded}
\begin{Highlighting}[]
\CommentTok{\#Astrid}

\CommentTok{\# making formulas for models with and without language as fixed effect}
\NormalTok{iqr\_m0 }\OtherTok{\textless{}{-}} \FunctionTok{bf}\NormalTok{(Pitch\_IQR\_stan }\SpecialCharTok{\textasciitilde{}} \DecValTok{1} \SpecialCharTok{+}\NormalTok{ Diagnosis }\SpecialCharTok{+}\NormalTok{ (}\DecValTok{1} \SpecialCharTok{|}\NormalTok{ ID))}
\NormalTok{iqr\_m1 }\OtherTok{\textless{}{-}} \FunctionTok{bf}\NormalTok{(Pitch\_IQR\_stan }\SpecialCharTok{\textasciitilde{}} \DecValTok{1} \SpecialCharTok{+}\NormalTok{ Diagnosis }\SpecialCharTok{+}\NormalTok{ Language }\SpecialCharTok{+}\NormalTok{ (}\DecValTok{1} \SpecialCharTok{|}\NormalTok{ ID))}

\CommentTok{\# getting priors for both}
\FunctionTok{get\_prior}\NormalTok{(iqr\_m0, data)}
\end{Highlighting}
\end{Shaded}

\begin{verbatim}
## Warning: Rows containing NAs were excluded from the model.
\end{verbatim}

\begin{verbatim}
##                    prior     class        coef group resp dpar nlpar bound
##                   (flat)         b                                        
##                   (flat)         b DiagnosisTD                            
##  student_t(3, -0.2, 2.5) Intercept                                        
##     student_t(3, 0, 2.5)        sd                                        
##     student_t(3, 0, 2.5)        sd                ID                      
##     student_t(3, 0, 2.5)        sd   Intercept    ID                      
##     student_t(3, 0, 2.5)     sigma                                        
##        source
##       default
##  (vectorized)
##       default
##       default
##  (vectorized)
##  (vectorized)
##       default
\end{verbatim}

\begin{Shaded}
\begin{Highlighting}[]
\FunctionTok{get\_prior}\NormalTok{(iqr\_m1, data)}
\end{Highlighting}
\end{Shaded}

\begin{verbatim}
## Warning: Rows containing NAs were excluded from the model.
\end{verbatim}

\begin{verbatim}
##                    prior     class        coef group resp dpar nlpar bound
##                   (flat)         b                                        
##                   (flat)         b DiagnosisTD                            
##                   (flat)         b  Languageus                            
##  student_t(3, -0.2, 2.5) Intercept                                        
##     student_t(3, 0, 2.5)        sd                                        
##     student_t(3, 0, 2.5)        sd                ID                      
##     student_t(3, 0, 2.5)        sd   Intercept    ID                      
##     student_t(3, 0, 2.5)     sigma                                        
##        source
##       default
##  (vectorized)
##  (vectorized)
##       default
##       default
##  (vectorized)
##  (vectorized)
##       default
\end{verbatim}

\begin{Shaded}
\begin{Highlighting}[]
\CommentTok{\# making priors for model without}
\NormalTok{m0\_pri }\OtherTok{\textless{}{-}} \FunctionTok{c}\NormalTok{(}
  \FunctionTok{prior}\NormalTok{(}\FunctionTok{normal}\NormalTok{(}\DecValTok{0}\NormalTok{, }\FloatTok{0.1}\NormalTok{), }\AttributeTok{class =}\NormalTok{ Intercept),}
  \FunctionTok{prior}\NormalTok{(}\FunctionTok{normal}\NormalTok{(}\DecValTok{1}\NormalTok{, }\FloatTok{0.5}\NormalTok{), }\AttributeTok{class =}\NormalTok{ sigma),}
  \FunctionTok{prior}\NormalTok{(}\FunctionTok{normal}\NormalTok{(}\DecValTok{0}\NormalTok{, }\FloatTok{0.13}\NormalTok{), }\AttributeTok{class =}\NormalTok{ b, }\AttributeTok{coef =}\NormalTok{ DiagnosisTD),}
  \FunctionTok{prior}\NormalTok{(}\FunctionTok{normal}\NormalTok{(}\DecValTok{0}\NormalTok{, .}\DecValTok{1}\NormalTok{), }\AttributeTok{class =}\NormalTok{ sd)}
\NormalTok{)}

\CommentTok{\# runnign model without language sampling from priors only}
\NormalTok{iqr\_m0\_spo }\OtherTok{\textless{}{-}} \FunctionTok{brm}\NormalTok{(}
  \AttributeTok{formula =}\NormalTok{ iqr\_m0,}
  \AttributeTok{data =}\NormalTok{ data,}
  \AttributeTok{family =}\NormalTok{ gaussian,}
  \AttributeTok{prior =}\NormalTok{ m0\_pri,}
  \AttributeTok{sample\_prior=}\StringTok{"only"}\NormalTok{,}
  \AttributeTok{chains =} \DecValTok{2}\NormalTok{,}
  \AttributeTok{cores =} \DecValTok{2}\NormalTok{,}
  \AttributeTok{backend =} \StringTok{"cmdstanr"}\NormalTok{,}
  \AttributeTok{threads =} \FunctionTok{threading}\NormalTok{(}\DecValTok{2}\NormalTok{)}
\NormalTok{)}
\end{Highlighting}
\end{Shaded}

\begin{verbatim}
## Warning: Rows containing NAs were excluded from the model.
\end{verbatim}

\begin{verbatim}
## Compiling Stan program...
\end{verbatim}

\begin{verbatim}
## In file included from stan/lib/stan_math/stan/math/rev/functor.hpp:28,
\end{verbatim}

\begin{verbatim}
##                  from stan/lib/stan_math/stan/math/rev.hpp:11,
\end{verbatim}

\begin{verbatim}
##                  from stan/lib/stan_math/stan/math.hpp:19,
\end{verbatim}

\begin{verbatim}
##                  from stan/src/stan/model/model_header.hpp:4,
\end{verbatim}

\begin{verbatim}
##                  from C:/Users/ukoto/AppData/Local/Temp/RtmpCcFj0B/model-206c18e52bbb.hpp:3:
\end{verbatim}

\begin{verbatim}
## stan/lib/stan_math/stan/math/rev/functor/reduce_sum.hpp: In instantiation of 'stan::math::internal::reduce_sum_impl<ReduceFunction, typename std::enable_if<stan::is_var<typename std::decay<_Arg>::type, void>::value, void>::type, ReturnType, Vec, Args ...>::recursive_reducer::recursive_reducer(size_t, size_t, double*, VecT&&, ArgsT&& ...) [with VecT = const std::vector<int>&; ArgsT = {const Eigen::Matrix<double, -1, 1, 0, -1, 1>&, const Eigen::Matrix<double, -1, -1, 0, -1, -1>&, Eigen::Matrix<stan::math::var_value<double, void>, -1, 1, 0, -1, 1>&, stan::math::var_value<double, void>&, stan::math::var_value<double, void>&, const std::vector<int, std::allocator<int> >&, const Eigen::Matrix<double, -1, 1, 0, -1, 1>&, Eigen::Matrix<stan::math::var_value<double, void>, -1, 1, 0, -1, 1>&}; ReduceFunction = file206c1b821b27_model_namespace::partial_log_lik_rsfunctor__; ReturnType = stan::math::var_value<double>; Vec = const std::vector<int>&; Args = {const Eigen::Matrix<double, -1, 1, 0, -1, 1>&, const Eigen::Matrix<double, -1, -1, 0, -1, -1>&, Eigen::Matrix<stan::math::var_value<double, void>, -1, 1, 0, -1, 1>&, stan::math::var_value<double, void>&, stan::math::var_value<double, void>&, const std::vector<int, std::allocator<int> >&, const Eigen::Matrix<double, -1, 1, 0, -1, 1>&, Eigen::Matrix<stan::math::var_value<double, void>, -1, 1, 0, -1, 1>&}; size_t = long long unsigned int]':
\end{verbatim}

\begin{verbatim}
## stan/lib/stan_math/stan/math/rev/functor/reduce_sum.hpp:248:23:   required from 'stan::math::var stan::math::internal::reduce_sum_impl<ReduceFunction, typename std::enable_if<stan::is_var<typename std::decay<_Arg>::type, void>::value, void>::type, ReturnType, Vec, Args ...>::operator()(Vec&&, bool, int, std::ostream*, Args&& ...) const [with ReduceFunction = file206c1b821b27_model_namespace::partial_log_lik_rsfunctor__; ReturnType = stan::math::var_value<double>; Vec = const std::vector<int>&; Args = {const Eigen::Matrix<double, -1, 1, 0, -1, 1>&, const Eigen::Matrix<double, -1, -1, 0, -1, -1>&, Eigen::Matrix<stan::math::var_value<double, void>, -1, 1, 0, -1, 1>&, stan::math::var_value<double, void>&, stan::math::var_value<double, void>&, const std::vector<int, std::allocator<int> >&, const Eigen::Matrix<double, -1, 1, 0, -1, 1>&, Eigen::Matrix<stan::math::var_value<double, void>, -1, 1, 0, -1, 1>&}; typename std::enable_if<stan::is_var<typename std::decay<_Arg>::type, void>::value, void>::type = void; stan::math::var = stan::math::var_value<double>; std::ostream = std::basic_ostream<char>]'
\end{verbatim}

\begin{verbatim}
## stan/lib/stan_math/stan/math/prim/functor/reduce_sum.hpp:207:46:   required from 'auto stan::math::reduce_sum(Vec&&, int, std::ostream*, Args&& ...) [with ReduceFunction = file206c1b821b27_model_namespace::partial_log_lik_rsfunctor__; Vec = const std::vector<int>&; <template-parameter-1-3> = void; Args = {const Eigen::Matrix<double, -1, 1, 0, -1, 1>&, const Eigen::Matrix<double, -1, -1, 0, -1, -1>&, Eigen::Matrix<stan::math::var_value<double, void>, -1, 1, 0, -1, 1>&, stan::math::var_value<double, void>&, stan::math::var_value<double, void>&, const std::vector<int, std::allocator<int> >&, const Eigen::Matrix<double, -1, 1, 0, -1, 1>&, Eigen::Matrix<stan::math::var_value<double, void>, -1, 1, 0, -1, 1>&}; std::ostream = std::basic_ostream<char>]'
\end{verbatim}

\begin{verbatim}
## C:/Users/ukoto/AppData/Local/Temp/RtmpCcFj0B/model-206c18e52bbb.hpp:627:52:   required from 'stan::scalar_type_t<T2> file206c1b821b27_model_namespace::file206c1b821b27_model::log_prob_impl(VecR&, VecI&, std::ostream*) const [with bool propto__ = false; bool jacobian__ = false; VecR = Eigen::Matrix<stan::math::var_value<double>, -1, 1>; VecI = Eigen::Matrix<int, -1, 1>; stan::require_vector_like_t<VecR>* <anonymous> = 0; stan::require_vector_like_vt<std::is_integral, VecI>* <anonymous> = 0; stan::scalar_type_t<T2> = stan::math::var_value<double>; std::ostream = std::basic_ostream<char>]'
\end{verbatim}

\begin{verbatim}
## C:/Users/ukoto/AppData/Local/Temp/RtmpCcFj0B/model-206c18e52bbb.hpp:1016:77:   required from 'T_ file206c1b821b27_model_namespace::file206c1b821b27_model::log_prob(Eigen::Matrix<T_job_param, -1, 1>&, std::ostream*) const [with bool propto__ = false; bool jacobian__ = false; T_ = stan::math::var_value<double>; std::ostream = std::basic_ostream<char>]'
\end{verbatim}

\begin{verbatim}
## stan/src/stan/model/model_base_crtp.hpp:96:77:   required from 'stan::math::var stan::model::model_base_crtp<M>::log_prob(Eigen::Matrix<stan::math::var_value<double>, -1, 1>&, std::ostream*) const [with M = file206c1b821b27_model_namespace::file206c1b821b27_model; stan::math::var = stan::math::var_value<double>; std::ostream = std::basic_ostream<char>]'
\end{verbatim}

\begin{verbatim}
## stan/src/stan/model/model_base_crtp.hpp:93:20:   required from here
\end{verbatim}

\begin{verbatim}
## stan/lib/stan_math/stan/math/rev/functor/reduce_sum.hpp:57:23: warning: 'stan::math::internal::reduce_sum_impl<file206c1b821b27_model_namespace::partial_log_lik_rsfunctor__, void, stan::math::var_value<double>, const std::vector<int>&, const Eigen::Matrix<double, -1, 1, 0, -1, 1>&, const Eigen::Matrix<double, -1, -1, 0, -1, -1>&, Eigen::Matrix<stan::math::var_value<double, void>, -1, 1, 0, -1, 1>&, stan::math::var_value<double, void>&, stan::math::var_value<double, void>&, const std::vector<int, std::allocator<int> >&, const Eigen::Matrix<double, -1, 1, 0, -1, 1>&, Eigen::Matrix<stan::math::var_value<double, void>, -1, 1, 0, -1, 1>&>::recursive_reducer::local_args_tuple_scope_' will be initialized after [-Wreorder]
\end{verbatim}

\begin{verbatim}
##      scoped_args_tuple local_args_tuple_scope_;
\end{verbatim}

\begin{verbatim}
##                        ^~~~~~~~~~~~~~~~~~~~~~~
\end{verbatim}

\begin{verbatim}
## stan/lib/stan_math/stan/math/rev/functor/reduce_sum.hpp:56:25: warning:   'std::tuple<const Eigen::Matrix<double, -1, 1, 0, -1, 1>&, const Eigen::Matrix<double, -1, -1, 0, -1, -1>&, Eigen::Matrix<stan::math::var_value<double, void>, -1, 1, 0, -1, 1>&, stan::math::var_value<double, void>&, stan::math::var_value<double, void>&, const std::vector<int, std::allocator<int> >&, const Eigen::Matrix<double, -1, 1, 0, -1, 1>&, Eigen::Matrix<stan::math::var_value<double, void>, -1, 1, 0, -1, 1>&> stan::math::internal::reduce_sum_impl<file206c1b821b27_model_namespace::partial_log_lik_rsfunctor__, void, stan::math::var_value<double>, const std::vector<int>&, const Eigen::Matrix<double, -1, 1, 0, -1, 1>&, const Eigen::Matrix<double, -1, -1, 0, -1, -1>&, Eigen::Matrix<stan::math::var_value<double, void>, -1, 1, 0, -1, 1>&, stan::math::var_value<double, void>&, stan::math::var_value<double, void>&, const std::vector<int, std::allocator<int> >&, const Eigen::Matrix<double, -1, 1, 0, -1, 1>&, Eigen::Matrix<stan::math::var_value<double, void>, -1, 1, 0, -1, 1>&>::recursive_reducer::args_tuple_' [-Wreorder]
\end{verbatim}

\begin{verbatim}
##      std::tuple<Args...> args_tuple_;
\end{verbatim}

\begin{verbatim}
##                          ^~~~~~~~~~~
\end{verbatim}

\begin{verbatim}
## stan/lib/stan_math/stan/math/rev/functor/reduce_sum.hpp:62:5: warning:   when initialized here [-Wreorder]
\end{verbatim}

\begin{verbatim}
##      recursive_reducer(size_t num_vars_per_term, size_t num_vars_shared_terms,
\end{verbatim}

\begin{verbatim}
##      ^~~~~~~~~~~~~~~~~
\end{verbatim}

\begin{verbatim}
## stan/lib/stan_math/stan/math/rev/functor/reduce_sum.hpp: In instantiation of 'stan::math::internal::reduce_sum_impl<ReduceFunction, typename std::enable_if<stan::is_var<typename std::decay<_Arg>::type, void>::value, void>::type, ReturnType, Vec, Args ...>::recursive_reducer::recursive_reducer(stan::math::internal::reduce_sum_impl<ReduceFunction, typename std::enable_if<stan::is_var<typename std::decay<_Arg>::type, void>::value, void>::type, ReturnType, Vec, Args ...>::recursive_reducer&, tbb::split) [with ReduceFunction = file206c1b821b27_model_namespace::partial_log_lik_rsfunctor__; ReturnType = stan::math::var_value<double>; Vec = const std::vector<int>&; Args = {const Eigen::Matrix<double, -1, 1, 0, -1, 1>&, const Eigen::Matrix<double, -1, -1, 0, -1, -1>&, Eigen::Matrix<stan::math::var_value<double, void>, -1, 1, 0, -1, 1>&, stan::math::var_value<double, void>&, stan::math::var_value<double, void>&, const std::vector<int, std::allocator<int> >&, const Eigen::Matrix<double, -1, 1, 0, -1, 1>&, Eigen::Matrix<stan::math::var_value<double, void>, -1, 1, 0, -1, 1>&}]':
\end{verbatim}

\begin{verbatim}
## stan/lib/stan_math/lib/tbb_2019_U8/include/tbb/parallel_reduce.h:183:27:   required from 'tbb::task* tbb::interface9::internal::start_reduce<Range, Body, Partitioner>::execute() [with Range = tbb::blocked_range<long long unsigned int>; Body = stan::math::internal::reduce_sum_impl<file206c1b821b27_model_namespace::partial_log_lik_rsfunctor__, void, stan::math::var_value<double>, const std::vector<int>&, const Eigen::Matrix<double, -1, 1, 0, -1, 1>&, const Eigen::Matrix<double, -1, -1, 0, -1, -1>&, Eigen::Matrix<stan::math::var_value<double, void>, -1, 1, 0, -1, 1>&, stan::math::var_value<double, void>&, stan::math::var_value<double, void>&, const std::vector<int, std::allocator<int> >&, const Eigen::Matrix<double, -1, 1, 0, -1, 1>&, Eigen::Matrix<stan::math::var_value<double, void>, -1, 1, 0, -1, 1>&>::recursive_reducer; Partitioner = const tbb::auto_partitioner]'
\end{verbatim}

\begin{verbatim}
## stan/lib/stan_math/lib/tbb_2019_U8/include/tbb/parallel_reduce.h:178:11:   required from here
\end{verbatim}

\begin{verbatim}
## stan/lib/stan_math/stan/math/rev/functor/reduce_sum.hpp:57:23: warning: 'stan::math::internal::reduce_sum_impl<file206c1b821b27_model_namespace::partial_log_lik_rsfunctor__, void, stan::math::var_value<double>, const std::vector<int>&, const Eigen::Matrix<double, -1, 1, 0, -1, 1>&, const Eigen::Matrix<double, -1, -1, 0, -1, -1>&, Eigen::Matrix<stan::math::var_value<double, void>, -1, 1, 0, -1, 1>&, stan::math::var_value<double, void>&, stan::math::var_value<double, void>&, const std::vector<int, std::allocator<int> >&, const Eigen::Matrix<double, -1, 1, 0, -1, 1>&, Eigen::Matrix<stan::math::var_value<double, void>, -1, 1, 0, -1, 1>&>::recursive_reducer::local_args_tuple_scope_' will be initialized after [-Wreorder]
\end{verbatim}

\begin{verbatim}
##      scoped_args_tuple local_args_tuple_scope_;
\end{verbatim}

\begin{verbatim}
##                        ^~~~~~~~~~~~~~~~~~~~~~~
\end{verbatim}

\begin{verbatim}
## stan/lib/stan_math/stan/math/rev/functor/reduce_sum.hpp:56:25: warning:   'std::tuple<const Eigen::Matrix<double, -1, 1, 0, -1, 1>&, const Eigen::Matrix<double, -1, -1, 0, -1, -1>&, Eigen::Matrix<stan::math::var_value<double, void>, -1, 1, 0, -1, 1>&, stan::math::var_value<double, void>&, stan::math::var_value<double, void>&, const std::vector<int, std::allocator<int> >&, const Eigen::Matrix<double, -1, 1, 0, -1, 1>&, Eigen::Matrix<stan::math::var_value<double, void>, -1, 1, 0, -1, 1>&> stan::math::internal::reduce_sum_impl<file206c1b821b27_model_namespace::partial_log_lik_rsfunctor__, void, stan::math::var_value<double>, const std::vector<int>&, const Eigen::Matrix<double, -1, 1, 0, -1, 1>&, const Eigen::Matrix<double, -1, -1, 0, -1, -1>&, Eigen::Matrix<stan::math::var_value<double, void>, -1, 1, 0, -1, 1>&, stan::math::var_value<double, void>&, stan::math::var_value<double, void>&, const std::vector<int, std::allocator<int> >&, const Eigen::Matrix<double, -1, 1, 0, -1, 1>&, Eigen::Matrix<stan::math::var_value<double, void>, -1, 1, 0, -1, 1>&>::recursive_reducer::args_tuple_' [-Wreorder]
\end{verbatim}

\begin{verbatim}
##      std::tuple<Args...> args_tuple_;
\end{verbatim}

\begin{verbatim}
##                          ^~~~~~~~~~~
\end{verbatim}

\begin{verbatim}
## stan/lib/stan_math/stan/math/rev/functor/reduce_sum.hpp:77:5: warning:   when initialized here [-Wreorder]
\end{verbatim}

\begin{verbatim}
##      recursive_reducer(recursive_reducer& other, tbb::split)
\end{verbatim}

\begin{verbatim}
##      ^~~~~~~~~~~~~~~~~
\end{verbatim}

\begin{verbatim}
## Start sampling
\end{verbatim}

\begin{verbatim}
## Running MCMC with 2 parallel chains, with 2 thread(s) per chain...
## 
## Chain 1 Iteration:    1 / 2000 [  0%]  (Warmup) 
## Chain 1 Iteration:  100 / 2000 [  5%]  (Warmup) 
## Chain 1 Iteration:  200 / 2000 [ 10%]  (Warmup) 
## Chain 2 Iteration:    1 / 2000 [  0%]  (Warmup) 
## Chain 2 Iteration:  100 / 2000 [  5%]  (Warmup) 
## Chain 1 Iteration:  300 / 2000 [ 15%]  (Warmup) 
## Chain 1 Iteration:  400 / 2000 [ 20%]  (Warmup) 
## Chain 1 Iteration:  500 / 2000 [ 25%]  (Warmup) 
## Chain 1 Iteration:  600 / 2000 [ 30%]  (Warmup) 
## Chain 1 Iteration:  700 / 2000 [ 35%]  (Warmup) 
## Chain 1 Iteration:  800 / 2000 [ 40%]  (Warmup) 
## Chain 1 Iteration:  900 / 2000 [ 45%]  (Warmup) 
## Chain 1 Iteration: 1000 / 2000 [ 50%]  (Warmup) 
## Chain 1 Iteration: 1001 / 2000 [ 50%]  (Sampling) 
## Chain 2 Iteration:  200 / 2000 [ 10%]  (Warmup) 
## Chain 2 Iteration:  300 / 2000 [ 15%]  (Warmup) 
## Chain 2 Iteration:  400 / 2000 [ 20%]  (Warmup) 
## Chain 2 Iteration:  500 / 2000 [ 25%]  (Warmup) 
## Chain 2 Iteration:  600 / 2000 [ 30%]  (Warmup) 
## Chain 2 Iteration:  700 / 2000 [ 35%]  (Warmup) 
## Chain 2 Iteration:  800 / 2000 [ 40%]  (Warmup) 
## Chain 2 Iteration:  900 / 2000 [ 45%]  (Warmup) 
## Chain 2 Iteration: 1000 / 2000 [ 50%]  (Warmup) 
## Chain 2 Iteration: 1001 / 2000 [ 50%]  (Sampling) 
## Chain 1 Iteration: 1100 / 2000 [ 55%]  (Sampling) 
## Chain 1 Iteration: 1200 / 2000 [ 60%]  (Sampling) 
## Chain 1 Iteration: 1300 / 2000 [ 65%]  (Sampling) 
## Chain 2 Iteration: 1100 / 2000 [ 55%]  (Sampling) 
## Chain 2 Iteration: 1200 / 2000 [ 60%]  (Sampling) 
## Chain 1 Iteration: 1400 / 2000 [ 70%]  (Sampling) 
## Chain 1 Iteration: 1500 / 2000 [ 75%]  (Sampling) 
## Chain 2 Iteration: 1300 / 2000 [ 65%]  (Sampling) 
## Chain 2 Iteration: 1400 / 2000 [ 70%]  (Sampling) 
## Chain 2 Iteration: 1500 / 2000 [ 75%]  (Sampling) 
## Chain 1 Iteration: 1600 / 2000 [ 80%]  (Sampling) 
## Chain 1 Iteration: 1700 / 2000 [ 85%]  (Sampling) 
## Chain 1 Iteration: 1800 / 2000 [ 90%]  (Sampling) 
## Chain 2 Iteration: 1600 / 2000 [ 80%]  (Sampling) 
## Chain 2 Iteration: 1700 / 2000 [ 85%]  (Sampling) 
## Chain 1 Iteration: 1900 / 2000 [ 95%]  (Sampling) 
## Chain 1 Iteration: 2000 / 2000 [100%]  (Sampling) 
## Chain 2 Iteration: 1800 / 2000 [ 90%]  (Sampling) 
## Chain 2 Iteration: 1900 / 2000 [ 95%]  (Sampling) 
## Chain 2 Iteration: 2000 / 2000 [100%]  (Sampling) 
## Chain 1 finished in 0.7 seconds.
## Chain 2 finished in 0.7 seconds.
## 
## Both chains finished successfully.
## Mean chain execution time: 0.7 seconds.
## Total execution time: 0.8 seconds.
\end{verbatim}

\begin{Shaded}
\begin{Highlighting}[]
\CommentTok{\# prior predictive check}
\FunctionTok{pp\_check}\NormalTok{(iqr\_m0\_spo, }\AttributeTok{nsample =} \DecValTok{100}\NormalTok{)}
\end{Highlighting}
\end{Shaded}

\includegraphics{Assignment3-final_files/figure-latex/unnamed-chunk-6-1.pdf}

\begin{Shaded}
\begin{Highlighting}[]
\CommentTok{\# running model now including data}
\NormalTok{iqr\_m0\_spt }\OtherTok{\textless{}{-}} \FunctionTok{brm}\NormalTok{(}
  \AttributeTok{formula =}\NormalTok{ iqr\_m0,}
  \AttributeTok{data =}\NormalTok{ data,}
  \AttributeTok{family =}\NormalTok{ gaussian,}
  \AttributeTok{prior =}\NormalTok{ m0\_pri,}
  \AttributeTok{sample\_prior=}\NormalTok{T,}
  \AttributeTok{chains =} \DecValTok{2}\NormalTok{,}
  \AttributeTok{cores =} \DecValTok{2}\NormalTok{,}
  \AttributeTok{backend =} \StringTok{"cmdstanr"}\NormalTok{,}
  \AttributeTok{threads =} \FunctionTok{threading}\NormalTok{(}\DecValTok{2}\NormalTok{)}
\NormalTok{)}
\end{Highlighting}
\end{Shaded}

\begin{verbatim}
## Warning: Rows containing NAs were excluded from the model.
\end{verbatim}

\begin{verbatim}
## Compiling Stan program...
\end{verbatim}

\begin{verbatim}
## In file included from stan/lib/stan_math/stan/math/rev/functor.hpp:28,
\end{verbatim}

\begin{verbatim}
##                  from stan/lib/stan_math/stan/math/rev.hpp:11,
\end{verbatim}

\begin{verbatim}
##                  from stan/lib/stan_math/stan/math.hpp:19,
\end{verbatim}

\begin{verbatim}
##                  from stan/src/stan/model/model_header.hpp:4,
\end{verbatim}

\begin{verbatim}
##                  from C:/Users/ukoto/AppData/Local/Temp/RtmpCcFj0B/model-206c79d3f11.hpp:3:
\end{verbatim}

\begin{verbatim}
## stan/lib/stan_math/stan/math/rev/functor/reduce_sum.hpp: In instantiation of 'stan::math::internal::reduce_sum_impl<ReduceFunction, typename std::enable_if<stan::is_var<typename std::decay<_Arg>::type, void>::value, void>::type, ReturnType, Vec, Args ...>::recursive_reducer::recursive_reducer(size_t, size_t, double*, VecT&&, ArgsT&& ...) [with VecT = const std::vector<int>&; ArgsT = {const Eigen::Matrix<double, -1, 1, 0, -1, 1>&, const Eigen::Matrix<double, -1, -1, 0, -1, -1>&, Eigen::Matrix<stan::math::var_value<double, void>, -1, 1, 0, -1, 1>&, stan::math::var_value<double, void>&, stan::math::var_value<double, void>&, const std::vector<int, std::allocator<int> >&, const Eigen::Matrix<double, -1, 1, 0, -1, 1>&, Eigen::Matrix<stan::math::var_value<double, void>, -1, 1, 0, -1, 1>&}; ReduceFunction = file206c30f84a1_model_namespace::partial_log_lik_rsfunctor__; ReturnType = stan::math::var_value<double>; Vec = const std::vector<int>&; Args = {const Eigen::Matrix<double, -1, 1, 0, -1, 1>&, const Eigen::Matrix<double, -1, -1, 0, -1, -1>&, Eigen::Matrix<stan::math::var_value<double, void>, -1, 1, 0, -1, 1>&, stan::math::var_value<double, void>&, stan::math::var_value<double, void>&, const std::vector<int, std::allocator<int> >&, const Eigen::Matrix<double, -1, 1, 0, -1, 1>&, Eigen::Matrix<stan::math::var_value<double, void>, -1, 1, 0, -1, 1>&}; size_t = long long unsigned int]':
\end{verbatim}

\begin{verbatim}
## stan/lib/stan_math/stan/math/rev/functor/reduce_sum.hpp:248:23:   required from 'stan::math::var stan::math::internal::reduce_sum_impl<ReduceFunction, typename std::enable_if<stan::is_var<typename std::decay<_Arg>::type, void>::value, void>::type, ReturnType, Vec, Args ...>::operator()(Vec&&, bool, int, std::ostream*, Args&& ...) const [with ReduceFunction = file206c30f84a1_model_namespace::partial_log_lik_rsfunctor__; ReturnType = stan::math::var_value<double>; Vec = const std::vector<int>&; Args = {const Eigen::Matrix<double, -1, 1, 0, -1, 1>&, const Eigen::Matrix<double, -1, -1, 0, -1, -1>&, Eigen::Matrix<stan::math::var_value<double, void>, -1, 1, 0, -1, 1>&, stan::math::var_value<double, void>&, stan::math::var_value<double, void>&, const std::vector<int, std::allocator<int> >&, const Eigen::Matrix<double, -1, 1, 0, -1, 1>&, Eigen::Matrix<stan::math::var_value<double, void>, -1, 1, 0, -1, 1>&}; typename std::enable_if<stan::is_var<typename std::decay<_Arg>::type, void>::value, void>::type = void; stan::math::var = stan::math::var_value<double>; std::ostream = std::basic_ostream<char>]'
\end{verbatim}

\begin{verbatim}
## stan/lib/stan_math/stan/math/prim/functor/reduce_sum.hpp:207:46:   required from 'auto stan::math::reduce_sum(Vec&&, int, std::ostream*, Args&& ...) [with ReduceFunction = file206c30f84a1_model_namespace::partial_log_lik_rsfunctor__; Vec = const std::vector<int>&; <template-parameter-1-3> = void; Args = {const Eigen::Matrix<double, -1, 1, 0, -1, 1>&, const Eigen::Matrix<double, -1, -1, 0, -1, -1>&, Eigen::Matrix<stan::math::var_value<double, void>, -1, 1, 0, -1, 1>&, stan::math::var_value<double, void>&, stan::math::var_value<double, void>&, const std::vector<int, std::allocator<int> >&, const Eigen::Matrix<double, -1, 1, 0, -1, 1>&, Eigen::Matrix<stan::math::var_value<double, void>, -1, 1, 0, -1, 1>&}; std::ostream = std::basic_ostream<char>]'
\end{verbatim}

\begin{verbatim}
## C:/Users/ukoto/AppData/Local/Temp/RtmpCcFj0B/model-206c79d3f11.hpp:637:52:   required from 'stan::scalar_type_t<T2> file206c30f84a1_model_namespace::file206c30f84a1_model::log_prob_impl(VecR&, VecI&, std::ostream*) const [with bool propto__ = false; bool jacobian__ = false; VecR = Eigen::Matrix<stan::math::var_value<double>, -1, 1>; VecI = Eigen::Matrix<int, -1, 1>; stan::require_vector_like_t<VecR>* <anonymous> = 0; stan::require_vector_like_vt<std::is_integral, VecI>* <anonymous> = 0; stan::scalar_type_t<T2> = stan::math::var_value<double>; std::ostream = std::basic_ostream<char>]'
\end{verbatim}

\begin{verbatim}
## C:/Users/ukoto/AppData/Local/Temp/RtmpCcFj0B/model-206c79d3f11.hpp:1080:77:   required from 'T_ file206c30f84a1_model_namespace::file206c30f84a1_model::log_prob(Eigen::Matrix<T_job_param, -1, 1>&, std::ostream*) const [with bool propto__ = false; bool jacobian__ = false; T_ = stan::math::var_value<double>; std::ostream = std::basic_ostream<char>]'
\end{verbatim}

\begin{verbatim}
## stan/src/stan/model/model_base_crtp.hpp:96:77:   required from 'stan::math::var stan::model::model_base_crtp<M>::log_prob(Eigen::Matrix<stan::math::var_value<double>, -1, 1>&, std::ostream*) const [with M = file206c30f84a1_model_namespace::file206c30f84a1_model; stan::math::var = stan::math::var_value<double>; std::ostream = std::basic_ostream<char>]'
\end{verbatim}

\begin{verbatim}
## stan/src/stan/model/model_base_crtp.hpp:93:20:   required from here
\end{verbatim}

\begin{verbatim}
## stan/lib/stan_math/stan/math/rev/functor/reduce_sum.hpp:57:23: warning: 'stan::math::internal::reduce_sum_impl<file206c30f84a1_model_namespace::partial_log_lik_rsfunctor__, void, stan::math::var_value<double>, const std::vector<int>&, const Eigen::Matrix<double, -1, 1, 0, -1, 1>&, const Eigen::Matrix<double, -1, -1, 0, -1, -1>&, Eigen::Matrix<stan::math::var_value<double, void>, -1, 1, 0, -1, 1>&, stan::math::var_value<double, void>&, stan::math::var_value<double, void>&, const std::vector<int, std::allocator<int> >&, const Eigen::Matrix<double, -1, 1, 0, -1, 1>&, Eigen::Matrix<stan::math::var_value<double, void>, -1, 1, 0, -1, 1>&>::recursive_reducer::local_args_tuple_scope_' will be initialized after [-Wreorder]
\end{verbatim}

\begin{verbatim}
##      scoped_args_tuple local_args_tuple_scope_;
\end{verbatim}

\begin{verbatim}
##                        ^~~~~~~~~~~~~~~~~~~~~~~
\end{verbatim}

\begin{verbatim}
## stan/lib/stan_math/stan/math/rev/functor/reduce_sum.hpp:56:25: warning:   'std::tuple<const Eigen::Matrix<double, -1, 1, 0, -1, 1>&, const Eigen::Matrix<double, -1, -1, 0, -1, -1>&, Eigen::Matrix<stan::math::var_value<double, void>, -1, 1, 0, -1, 1>&, stan::math::var_value<double, void>&, stan::math::var_value<double, void>&, const std::vector<int, std::allocator<int> >&, const Eigen::Matrix<double, -1, 1, 0, -1, 1>&, Eigen::Matrix<stan::math::var_value<double, void>, -1, 1, 0, -1, 1>&> stan::math::internal::reduce_sum_impl<file206c30f84a1_model_namespace::partial_log_lik_rsfunctor__, void, stan::math::var_value<double>, const std::vector<int>&, const Eigen::Matrix<double, -1, 1, 0, -1, 1>&, const Eigen::Matrix<double, -1, -1, 0, -1, -1>&, Eigen::Matrix<stan::math::var_value<double, void>, -1, 1, 0, -1, 1>&, stan::math::var_value<double, void>&, stan::math::var_value<double, void>&, const std::vector<int, std::allocator<int> >&, const Eigen::Matrix<double, -1, 1, 0, -1, 1>&, Eigen::Matrix<stan::math::var_value<double, void>, -1, 1, 0, -1, 1>&>::recursive_reducer::args_tuple_' [-Wreorder]
\end{verbatim}

\begin{verbatim}
##      std::tuple<Args...> args_tuple_;
\end{verbatim}

\begin{verbatim}
##                          ^~~~~~~~~~~
\end{verbatim}

\begin{verbatim}
## stan/lib/stan_math/stan/math/rev/functor/reduce_sum.hpp:62:5: warning:   when initialized here [-Wreorder]
\end{verbatim}

\begin{verbatim}
##      recursive_reducer(size_t num_vars_per_term, size_t num_vars_shared_terms,
\end{verbatim}

\begin{verbatim}
##      ^~~~~~~~~~~~~~~~~
\end{verbatim}

\begin{verbatim}
## stan/lib/stan_math/stan/math/rev/functor/reduce_sum.hpp: In instantiation of 'stan::math::internal::reduce_sum_impl<ReduceFunction, typename std::enable_if<stan::is_var<typename std::decay<_Arg>::type, void>::value, void>::type, ReturnType, Vec, Args ...>::recursive_reducer::recursive_reducer(stan::math::internal::reduce_sum_impl<ReduceFunction, typename std::enable_if<stan::is_var<typename std::decay<_Arg>::type, void>::value, void>::type, ReturnType, Vec, Args ...>::recursive_reducer&, tbb::split) [with ReduceFunction = file206c30f84a1_model_namespace::partial_log_lik_rsfunctor__; ReturnType = stan::math::var_value<double>; Vec = const std::vector<int>&; Args = {const Eigen::Matrix<double, -1, 1, 0, -1, 1>&, const Eigen::Matrix<double, -1, -1, 0, -1, -1>&, Eigen::Matrix<stan::math::var_value<double, void>, -1, 1, 0, -1, 1>&, stan::math::var_value<double, void>&, stan::math::var_value<double, void>&, const std::vector<int, std::allocator<int> >&, const Eigen::Matrix<double, -1, 1, 0, -1, 1>&, Eigen::Matrix<stan::math::var_value<double, void>, -1, 1, 0, -1, 1>&}]':
\end{verbatim}

\begin{verbatim}
## stan/lib/stan_math/lib/tbb_2019_U8/include/tbb/parallel_reduce.h:183:27:   required from 'tbb::task* tbb::interface9::internal::start_reduce<Range, Body, Partitioner>::execute() [with Range = tbb::blocked_range<long long unsigned int>; Body = stan::math::internal::reduce_sum_impl<file206c30f84a1_model_namespace::partial_log_lik_rsfunctor__, void, stan::math::var_value<double>, const std::vector<int>&, const Eigen::Matrix<double, -1, 1, 0, -1, 1>&, const Eigen::Matrix<double, -1, -1, 0, -1, -1>&, Eigen::Matrix<stan::math::var_value<double, void>, -1, 1, 0, -1, 1>&, stan::math::var_value<double, void>&, stan::math::var_value<double, void>&, const std::vector<int, std::allocator<int> >&, const Eigen::Matrix<double, -1, 1, 0, -1, 1>&, Eigen::Matrix<stan::math::var_value<double, void>, -1, 1, 0, -1, 1>&>::recursive_reducer; Partitioner = const tbb::auto_partitioner]'
\end{verbatim}

\begin{verbatim}
## stan/lib/stan_math/lib/tbb_2019_U8/include/tbb/parallel_reduce.h:178:11:   required from here
\end{verbatim}

\begin{verbatim}
## stan/lib/stan_math/stan/math/rev/functor/reduce_sum.hpp:57:23: warning: 'stan::math::internal::reduce_sum_impl<file206c30f84a1_model_namespace::partial_log_lik_rsfunctor__, void, stan::math::var_value<double>, const std::vector<int>&, const Eigen::Matrix<double, -1, 1, 0, -1, 1>&, const Eigen::Matrix<double, -1, -1, 0, -1, -1>&, Eigen::Matrix<stan::math::var_value<double, void>, -1, 1, 0, -1, 1>&, stan::math::var_value<double, void>&, stan::math::var_value<double, void>&, const std::vector<int, std::allocator<int> >&, const Eigen::Matrix<double, -1, 1, 0, -1, 1>&, Eigen::Matrix<stan::math::var_value<double, void>, -1, 1, 0, -1, 1>&>::recursive_reducer::local_args_tuple_scope_' will be initialized after [-Wreorder]
\end{verbatim}

\begin{verbatim}
##      scoped_args_tuple local_args_tuple_scope_;
\end{verbatim}

\begin{verbatim}
##                        ^~~~~~~~~~~~~~~~~~~~~~~
\end{verbatim}

\begin{verbatim}
## stan/lib/stan_math/stan/math/rev/functor/reduce_sum.hpp:56:25: warning:   'std::tuple<const Eigen::Matrix<double, -1, 1, 0, -1, 1>&, const Eigen::Matrix<double, -1, -1, 0, -1, -1>&, Eigen::Matrix<stan::math::var_value<double, void>, -1, 1, 0, -1, 1>&, stan::math::var_value<double, void>&, stan::math::var_value<double, void>&, const std::vector<int, std::allocator<int> >&, const Eigen::Matrix<double, -1, 1, 0, -1, 1>&, Eigen::Matrix<stan::math::var_value<double, void>, -1, 1, 0, -1, 1>&> stan::math::internal::reduce_sum_impl<file206c30f84a1_model_namespace::partial_log_lik_rsfunctor__, void, stan::math::var_value<double>, const std::vector<int>&, const Eigen::Matrix<double, -1, 1, 0, -1, 1>&, const Eigen::Matrix<double, -1, -1, 0, -1, -1>&, Eigen::Matrix<stan::math::var_value<double, void>, -1, 1, 0, -1, 1>&, stan::math::var_value<double, void>&, stan::math::var_value<double, void>&, const std::vector<int, std::allocator<int> >&, const Eigen::Matrix<double, -1, 1, 0, -1, 1>&, Eigen::Matrix<stan::math::var_value<double, void>, -1, 1, 0, -1, 1>&>::recursive_reducer::args_tuple_' [-Wreorder]
\end{verbatim}

\begin{verbatim}
##      std::tuple<Args...> args_tuple_;
\end{verbatim}

\begin{verbatim}
##                          ^~~~~~~~~~~
\end{verbatim}

\begin{verbatim}
## stan/lib/stan_math/stan/math/rev/functor/reduce_sum.hpp:77:5: warning:   when initialized here [-Wreorder]
\end{verbatim}

\begin{verbatim}
##      recursive_reducer(recursive_reducer& other, tbb::split)
\end{verbatim}

\begin{verbatim}
##      ^~~~~~~~~~~~~~~~~
\end{verbatim}

\begin{verbatim}
## Start sampling
\end{verbatim}

\begin{verbatim}
## Running MCMC with 2 parallel chains, with 2 thread(s) per chain...
## 
## Chain 1 Iteration:    1 / 2000 [  0%]  (Warmup) 
## Chain 2 Iteration:    1 / 2000 [  0%]  (Warmup) 
## Chain 1 Iteration:  100 / 2000 [  5%]  (Warmup) 
## Chain 1 Iteration:  200 / 2000 [ 10%]  (Warmup) 
## Chain 1 Iteration:  300 / 2000 [ 15%]  (Warmup) 
## Chain 2 Iteration:  100 / 2000 [  5%]  (Warmup) 
## Chain 1 Iteration:  400 / 2000 [ 20%]  (Warmup) 
## Chain 2 Iteration:  200 / 2000 [ 10%]  (Warmup) 
## Chain 2 Iteration:  300 / 2000 [ 15%]  (Warmup) 
## Chain 1 Iteration:  500 / 2000 [ 25%]  (Warmup) 
## Chain 2 Iteration:  400 / 2000 [ 20%]  (Warmup) 
## Chain 1 Iteration:  600 / 2000 [ 30%]  (Warmup) 
## Chain 1 Iteration:  700 / 2000 [ 35%]  (Warmup) 
## Chain 2 Iteration:  500 / 2000 [ 25%]  (Warmup) 
## Chain 1 Iteration:  800 / 2000 [ 40%]  (Warmup) 
## Chain 2 Iteration:  600 / 2000 [ 30%]  (Warmup) 
## Chain 2 Iteration:  700 / 2000 [ 35%]  (Warmup) 
## Chain 1 Iteration:  900 / 2000 [ 45%]  (Warmup) 
## Chain 2 Iteration:  800 / 2000 [ 40%]  (Warmup) 
## Chain 1 Iteration: 1000 / 2000 [ 50%]  (Warmup) 
## Chain 1 Iteration: 1001 / 2000 [ 50%]  (Sampling) 
## Chain 2 Iteration:  900 / 2000 [ 45%]  (Warmup) 
## Chain 2 Iteration: 1000 / 2000 [ 50%]  (Warmup) 
## Chain 2 Iteration: 1001 / 2000 [ 50%]  (Sampling) 
## Chain 1 Iteration: 1100 / 2000 [ 55%]  (Sampling) 
## Chain 2 Iteration: 1100 / 2000 [ 55%]  (Sampling) 
## Chain 1 Iteration: 1200 / 2000 [ 60%]  (Sampling) 
## Chain 1 Iteration: 1300 / 2000 [ 65%]  (Sampling) 
## Chain 2 Iteration: 1200 / 2000 [ 60%]  (Sampling) 
## Chain 1 Iteration: 1400 / 2000 [ 70%]  (Sampling) 
## Chain 2 Iteration: 1300 / 2000 [ 65%]  (Sampling) 
## Chain 1 Iteration: 1500 / 2000 [ 75%]  (Sampling) 
## Chain 2 Iteration: 1400 / 2000 [ 70%]  (Sampling) 
## Chain 1 Iteration: 1600 / 2000 [ 80%]  (Sampling) 
## Chain 2 Iteration: 1500 / 2000 [ 75%]  (Sampling) 
## Chain 1 Iteration: 1700 / 2000 [ 85%]  (Sampling) 
## Chain 2 Iteration: 1600 / 2000 [ 80%]  (Sampling) 
## Chain 1 Iteration: 1800 / 2000 [ 90%]  (Sampling) 
## Chain 2 Iteration: 1700 / 2000 [ 85%]  (Sampling) 
## Chain 1 Iteration: 1900 / 2000 [ 95%]  (Sampling) 
## Chain 1 Iteration: 2000 / 2000 [100%]  (Sampling) 
## Chain 2 Iteration: 1800 / 2000 [ 90%]  (Sampling) 
## Chain 1 finished in 2.5 seconds.
## Chain 2 Iteration: 1900 / 2000 [ 95%]  (Sampling) 
## Chain 2 Iteration: 2000 / 2000 [100%]  (Sampling) 
## Chain 2 finished in 2.6 seconds.
## 
## Both chains finished successfully.
## Mean chain execution time: 2.5 seconds.
## Total execution time: 2.8 seconds.
\end{verbatim}

\begin{Shaded}
\begin{Highlighting}[]
\CommentTok{\# posterior predictive check}
\FunctionTok{pp\_check}\NormalTok{(iqr\_m0\_spt, }\AttributeTok{nsample =} \DecValTok{100}\NormalTok{)}
\end{Highlighting}
\end{Shaded}

\includegraphics{Assignment3-final_files/figure-latex/unnamed-chunk-6-2.pdf}

\begin{Shaded}
\begin{Highlighting}[]
\CommentTok{\# checking effect of diagnosis}
\FunctionTok{hypothesis}\NormalTok{(iqr\_m0\_spt, }\StringTok{"DiagnosisTD \textless{} 0"}\NormalTok{)}
\end{Highlighting}
\end{Shaded}

\begin{verbatim}
## Hypothesis Tests for class b:
##          Hypothesis Estimate Est.Error CI.Lower CI.Upper Evid.Ratio Post.Prob
## 1 (DiagnosisTD) < 0    -0.13      0.09    -0.27     0.02      11.66      0.92
##   Star
## 1     
## ---
## 'CI': 90%-CI for one-sided and 95%-CI for two-sided hypotheses.
## '*': For one-sided hypotheses, the posterior probability exceeds 95%;
## for two-sided hypotheses, the value tested against lies outside the 95%-CI.
## Posterior probabilities of point hypotheses assume equal prior probabilities.
\end{verbatim}

\begin{Shaded}
\begin{Highlighting}[]
\FunctionTok{plot}\NormalTok{(}\FunctionTok{hypothesis}\NormalTok{(iqr\_m0\_spt, }\StringTok{"DiagnosisTD \textless{} 0"}\NormalTok{))}
\end{Highlighting}
\end{Shaded}

\includegraphics{Assignment3-final_files/figure-latex/unnamed-chunk-6-3.pdf}

\begin{Shaded}
\begin{Highlighting}[]
\CommentTok{\# summary of model}
\FunctionTok{summary}\NormalTok{(iqr\_m0\_spt)}
\end{Highlighting}
\end{Shaded}

\begin{verbatim}
##  Family: gaussian 
##   Links: mu = identity; sigma = identity 
## Formula: Pitch_IQR_stan ~ 1 + Diagnosis + (1 | ID) 
##    Data: data (Number of observations: 1042) 
## Samples: 2 chains, each with iter = 2000; warmup = 1000; thin = 1;
##          total post-warmup samples = 2000
## 
## Group-Level Effects: 
## ~ID (Number of levels: 140) 
##               Estimate Est.Error l-95% CI u-95% CI Rhat Bulk_ESS Tail_ESS
## sd(Intercept)     0.63      0.04     0.56     0.70 1.00      655     1021
## 
## Population-Level Effects: 
##             Estimate Est.Error l-95% CI u-95% CI Rhat Bulk_ESS Tail_ESS
## Intercept       0.20      0.07     0.06     0.33 1.00      764     1140
## DiagnosisTD    -0.13      0.09    -0.30     0.05 1.00      996     1222
## 
## Family Specific Parameters: 
##       Estimate Est.Error l-95% CI u-95% CI Rhat Bulk_ESS Tail_ESS
## sigma     0.71      0.02     0.68     0.75 1.00     2417     1667
## 
## Samples were drawn using sample(hmc). For each parameter, Bulk_ESS
## and Tail_ESS are effective sample size measures, and Rhat is the potential
## scale reduction factor on split chains (at convergence, Rhat = 1).
\end{verbatim}

\#\#\#\#Step 4: Now re-run the model with the meta-analytic prior -
Evaluate model quality. Describe and plot the estimates. - N.B. you need
to assess the meta-analytic informed prior (prior pred checks,
prior-posterior update checks) and if relevant you can always change it
in motivated ways (e.g.~too confident, doesn't let the model actually
learn from the data, so increase sd)

\begin{Shaded}
\begin{Highlighting}[]
\CommentTok{\#Tobias}
\CommentTok{\# getting mean, effect size and standard error for meta priors}
\NormalTok{meta\_mean}
\end{Highlighting}
\end{Shaded}

\begin{verbatim}
## [1] -0.4394258
\end{verbatim}

\begin{Shaded}
\begin{Highlighting}[]
\NormalTok{meta\_error}
\end{Highlighting}
\end{Shaded}

\begin{verbatim}
## [1] 0.09740861
\end{verbatim}

\begin{Shaded}
\begin{Highlighting}[]
\NormalTok{meta\_effectsize}
\end{Highlighting}
\end{Shaded}

\begin{verbatim}
## [1] 0.32
\end{verbatim}

\begin{Shaded}
\begin{Highlighting}[]
\CommentTok{\# making meta priors}
\NormalTok{meta\_pri }\OtherTok{\textless{}{-}} \FunctionTok{c}\NormalTok{(}
  \FunctionTok{prior}\NormalTok{(}\FunctionTok{normal}\NormalTok{(}\DecValTok{0}\NormalTok{, }\FloatTok{0.1}\NormalTok{), }\AttributeTok{class =}\NormalTok{ Intercept),}
  \FunctionTok{prior}\NormalTok{(}\FunctionTok{normal}\NormalTok{(.}\DecValTok{32}\NormalTok{, }\FloatTok{0.5}\NormalTok{), }\AttributeTok{class =}\NormalTok{ sigma), }\CommentTok{\# corresponds to the effect size from meta analysis}
  \FunctionTok{prior}\NormalTok{(}\FunctionTok{normal}\NormalTok{(}\SpecialCharTok{{-}}\FloatTok{0.436}\NormalTok{, }\FloatTok{0.09}\NormalTok{), }\AttributeTok{class =}\NormalTok{ b, }\AttributeTok{coef =}\NormalTok{ DiagnosisTD), }\CommentTok{\# corresponds to mean and standard error from meta analysis}
  \FunctionTok{prior}\NormalTok{(}\FunctionTok{normal}\NormalTok{(}\DecValTok{0}\NormalTok{, .}\DecValTok{1}\NormalTok{), }\AttributeTok{class =}\NormalTok{ sd)}
\NormalTok{)}

\CommentTok{\# run model with meta priors}
\NormalTok{iqr\_meta\_spo }\OtherTok{\textless{}{-}} \FunctionTok{brm}\NormalTok{(}
  \AttributeTok{formula =}\NormalTok{ iqr\_m0,}
  \AttributeTok{data =}\NormalTok{ data,}
  \AttributeTok{family =}\NormalTok{ gaussian,}
  \AttributeTok{prior =}\NormalTok{ meta\_pri,}
  \AttributeTok{sample\_prior =} \StringTok{"only"}\NormalTok{,}
  \AttributeTok{chains =} \DecValTok{2}\NormalTok{,}
  \AttributeTok{cores =} \DecValTok{2}\NormalTok{,}
  \AttributeTok{backend =} \StringTok{"cmdstanr"}\NormalTok{,}
  \AttributeTok{threads =} \FunctionTok{threading}\NormalTok{(}\DecValTok{2}\NormalTok{)}
\NormalTok{)}
\end{Highlighting}
\end{Shaded}

\begin{verbatim}
## Warning: Rows containing NAs were excluded from the model.
\end{verbatim}

\begin{verbatim}
## Compiling Stan program...
\end{verbatim}

\begin{verbatim}
## In file included from stan/lib/stan_math/stan/math/rev/functor.hpp:28,
\end{verbatim}

\begin{verbatim}
##                  from stan/lib/stan_math/stan/math/rev.hpp:11,
\end{verbatim}

\begin{verbatim}
##                  from stan/lib/stan_math/stan/math.hpp:19,
\end{verbatim}

\begin{verbatim}
##                  from stan/src/stan/model/model_header.hpp:4,
\end{verbatim}

\begin{verbatim}
##                  from C:/Users/ukoto/AppData/Local/Temp/RtmpCcFj0B/model-206c403d30c4.hpp:3:
\end{verbatim}

\begin{verbatim}
## stan/lib/stan_math/stan/math/rev/functor/reduce_sum.hpp: In instantiation of 'stan::math::internal::reduce_sum_impl<ReduceFunction, typename std::enable_if<stan::is_var<typename std::decay<_Arg>::type, void>::value, void>::type, ReturnType, Vec, Args ...>::recursive_reducer::recursive_reducer(size_t, size_t, double*, VecT&&, ArgsT&& ...) [with VecT = const std::vector<int>&; ArgsT = {const Eigen::Matrix<double, -1, 1, 0, -1, 1>&, const Eigen::Matrix<double, -1, -1, 0, -1, -1>&, Eigen::Matrix<stan::math::var_value<double, void>, -1, 1, 0, -1, 1>&, stan::math::var_value<double, void>&, stan::math::var_value<double, void>&, const std::vector<int, std::allocator<int> >&, const Eigen::Matrix<double, -1, 1, 0, -1, 1>&, Eigen::Matrix<stan::math::var_value<double, void>, -1, 1, 0, -1, 1>&}; ReduceFunction = file206c10244950_model_namespace::partial_log_lik_rsfunctor__; ReturnType = stan::math::var_value<double>; Vec = const std::vector<int>&; Args = {const Eigen::Matrix<double, -1, 1, 0, -1, 1>&, const Eigen::Matrix<double, -1, -1, 0, -1, -1>&, Eigen::Matrix<stan::math::var_value<double, void>, -1, 1, 0, -1, 1>&, stan::math::var_value<double, void>&, stan::math::var_value<double, void>&, const std::vector<int, std::allocator<int> >&, const Eigen::Matrix<double, -1, 1, 0, -1, 1>&, Eigen::Matrix<stan::math::var_value<double, void>, -1, 1, 0, -1, 1>&}; size_t = long long unsigned int]':
\end{verbatim}

\begin{verbatim}
## stan/lib/stan_math/stan/math/rev/functor/reduce_sum.hpp:248:23:   required from 'stan::math::var stan::math::internal::reduce_sum_impl<ReduceFunction, typename std::enable_if<stan::is_var<typename std::decay<_Arg>::type, void>::value, void>::type, ReturnType, Vec, Args ...>::operator()(Vec&&, bool, int, std::ostream*, Args&& ...) const [with ReduceFunction = file206c10244950_model_namespace::partial_log_lik_rsfunctor__; ReturnType = stan::math::var_value<double>; Vec = const std::vector<int>&; Args = {const Eigen::Matrix<double, -1, 1, 0, -1, 1>&, const Eigen::Matrix<double, -1, -1, 0, -1, -1>&, Eigen::Matrix<stan::math::var_value<double, void>, -1, 1, 0, -1, 1>&, stan::math::var_value<double, void>&, stan::math::var_value<double, void>&, const std::vector<int, std::allocator<int> >&, const Eigen::Matrix<double, -1, 1, 0, -1, 1>&, Eigen::Matrix<stan::math::var_value<double, void>, -1, 1, 0, -1, 1>&}; typename std::enable_if<stan::is_var<typename std::decay<_Arg>::type, void>::value, void>::type = void; stan::math::var = stan::math::var_value<double>; std::ostream = std::basic_ostream<char>]'
\end{verbatim}

\begin{verbatim}
## stan/lib/stan_math/stan/math/prim/functor/reduce_sum.hpp:207:46:   required from 'auto stan::math::reduce_sum(Vec&&, int, std::ostream*, Args&& ...) [with ReduceFunction = file206c10244950_model_namespace::partial_log_lik_rsfunctor__; Vec = const std::vector<int>&; <template-parameter-1-3> = void; Args = {const Eigen::Matrix<double, -1, 1, 0, -1, 1>&, const Eigen::Matrix<double, -1, -1, 0, -1, -1>&, Eigen::Matrix<stan::math::var_value<double, void>, -1, 1, 0, -1, 1>&, stan::math::var_value<double, void>&, stan::math::var_value<double, void>&, const std::vector<int, std::allocator<int> >&, const Eigen::Matrix<double, -1, 1, 0, -1, 1>&, Eigen::Matrix<stan::math::var_value<double, void>, -1, 1, 0, -1, 1>&}; std::ostream = std::basic_ostream<char>]'
\end{verbatim}

\begin{verbatim}
## C:/Users/ukoto/AppData/Local/Temp/RtmpCcFj0B/model-206c403d30c4.hpp:627:52:   required from 'stan::scalar_type_t<T2> file206c10244950_model_namespace::file206c10244950_model::log_prob_impl(VecR&, VecI&, std::ostream*) const [with bool propto__ = false; bool jacobian__ = false; VecR = Eigen::Matrix<stan::math::var_value<double>, -1, 1>; VecI = Eigen::Matrix<int, -1, 1>; stan::require_vector_like_t<VecR>* <anonymous> = 0; stan::require_vector_like_vt<std::is_integral, VecI>* <anonymous> = 0; stan::scalar_type_t<T2> = stan::math::var_value<double>; std::ostream = std::basic_ostream<char>]'
\end{verbatim}

\begin{verbatim}
## C:/Users/ukoto/AppData/Local/Temp/RtmpCcFj0B/model-206c403d30c4.hpp:1017:77:   required from 'T_ file206c10244950_model_namespace::file206c10244950_model::log_prob(Eigen::Matrix<T_job_param, -1, 1>&, std::ostream*) const [with bool propto__ = false; bool jacobian__ = false; T_ = stan::math::var_value<double>; std::ostream = std::basic_ostream<char>]'
\end{verbatim}

\begin{verbatim}
## stan/src/stan/model/model_base_crtp.hpp:96:77:   required from 'stan::math::var stan::model::model_base_crtp<M>::log_prob(Eigen::Matrix<stan::math::var_value<double>, -1, 1>&, std::ostream*) const [with M = file206c10244950_model_namespace::file206c10244950_model; stan::math::var = stan::math::var_value<double>; std::ostream = std::basic_ostream<char>]'
\end{verbatim}

\begin{verbatim}
## stan/src/stan/model/model_base_crtp.hpp:93:20:   required from here
\end{verbatim}

\begin{verbatim}
## stan/lib/stan_math/stan/math/rev/functor/reduce_sum.hpp:57:23: warning: 'stan::math::internal::reduce_sum_impl<file206c10244950_model_namespace::partial_log_lik_rsfunctor__, void, stan::math::var_value<double>, const std::vector<int>&, const Eigen::Matrix<double, -1, 1, 0, -1, 1>&, const Eigen::Matrix<double, -1, -1, 0, -1, -1>&, Eigen::Matrix<stan::math::var_value<double, void>, -1, 1, 0, -1, 1>&, stan::math::var_value<double, void>&, stan::math::var_value<double, void>&, const std::vector<int, std::allocator<int> >&, const Eigen::Matrix<double, -1, 1, 0, -1, 1>&, Eigen::Matrix<stan::math::var_value<double, void>, -1, 1, 0, -1, 1>&>::recursive_reducer::local_args_tuple_scope_' will be initialized after [-Wreorder]
\end{verbatim}

\begin{verbatim}
##      scoped_args_tuple local_args_tuple_scope_;
\end{verbatim}

\begin{verbatim}
##                        ^~~~~~~~~~~~~~~~~~~~~~~
\end{verbatim}

\begin{verbatim}
## stan/lib/stan_math/stan/math/rev/functor/reduce_sum.hpp:56:25: warning:   'std::tuple<const Eigen::Matrix<double, -1, 1, 0, -1, 1>&, const Eigen::Matrix<double, -1, -1, 0, -1, -1>&, Eigen::Matrix<stan::math::var_value<double, void>, -1, 1, 0, -1, 1>&, stan::math::var_value<double, void>&, stan::math::var_value<double, void>&, const std::vector<int, std::allocator<int> >&, const Eigen::Matrix<double, -1, 1, 0, -1, 1>&, Eigen::Matrix<stan::math::var_value<double, void>, -1, 1, 0, -1, 1>&> stan::math::internal::reduce_sum_impl<file206c10244950_model_namespace::partial_log_lik_rsfunctor__, void, stan::math::var_value<double>, const std::vector<int>&, const Eigen::Matrix<double, -1, 1, 0, -1, 1>&, const Eigen::Matrix<double, -1, -1, 0, -1, -1>&, Eigen::Matrix<stan::math::var_value<double, void>, -1, 1, 0, -1, 1>&, stan::math::var_value<double, void>&, stan::math::var_value<double, void>&, const std::vector<int, std::allocator<int> >&, const Eigen::Matrix<double, -1, 1, 0, -1, 1>&, Eigen::Matrix<stan::math::var_value<double, void>, -1, 1, 0, -1, 1>&>::recursive_reducer::args_tuple_' [-Wreorder]
\end{verbatim}

\begin{verbatim}
##      std::tuple<Args...> args_tuple_;
\end{verbatim}

\begin{verbatim}
##                          ^~~~~~~~~~~
\end{verbatim}

\begin{verbatim}
## stan/lib/stan_math/stan/math/rev/functor/reduce_sum.hpp:62:5: warning:   when initialized here [-Wreorder]
\end{verbatim}

\begin{verbatim}
##      recursive_reducer(size_t num_vars_per_term, size_t num_vars_shared_terms,
\end{verbatim}

\begin{verbatim}
##      ^~~~~~~~~~~~~~~~~
\end{verbatim}

\begin{verbatim}
## stan/lib/stan_math/stan/math/rev/functor/reduce_sum.hpp: In instantiation of 'stan::math::internal::reduce_sum_impl<ReduceFunction, typename std::enable_if<stan::is_var<typename std::decay<_Arg>::type, void>::value, void>::type, ReturnType, Vec, Args ...>::recursive_reducer::recursive_reducer(stan::math::internal::reduce_sum_impl<ReduceFunction, typename std::enable_if<stan::is_var<typename std::decay<_Arg>::type, void>::value, void>::type, ReturnType, Vec, Args ...>::recursive_reducer&, tbb::split) [with ReduceFunction = file206c10244950_model_namespace::partial_log_lik_rsfunctor__; ReturnType = stan::math::var_value<double>; Vec = const std::vector<int>&; Args = {const Eigen::Matrix<double, -1, 1, 0, -1, 1>&, const Eigen::Matrix<double, -1, -1, 0, -1, -1>&, Eigen::Matrix<stan::math::var_value<double, void>, -1, 1, 0, -1, 1>&, stan::math::var_value<double, void>&, stan::math::var_value<double, void>&, const std::vector<int, std::allocator<int> >&, const Eigen::Matrix<double, -1, 1, 0, -1, 1>&, Eigen::Matrix<stan::math::var_value<double, void>, -1, 1, 0, -1, 1>&}]':
\end{verbatim}

\begin{verbatim}
## stan/lib/stan_math/lib/tbb_2019_U8/include/tbb/parallel_reduce.h:183:27:   required from 'tbb::task* tbb::interface9::internal::start_reduce<Range, Body, Partitioner>::execute() [with Range = tbb::blocked_range<long long unsigned int>; Body = stan::math::internal::reduce_sum_impl<file206c10244950_model_namespace::partial_log_lik_rsfunctor__, void, stan::math::var_value<double>, const std::vector<int>&, const Eigen::Matrix<double, -1, 1, 0, -1, 1>&, const Eigen::Matrix<double, -1, -1, 0, -1, -1>&, Eigen::Matrix<stan::math::var_value<double, void>, -1, 1, 0, -1, 1>&, stan::math::var_value<double, void>&, stan::math::var_value<double, void>&, const std::vector<int, std::allocator<int> >&, const Eigen::Matrix<double, -1, 1, 0, -1, 1>&, Eigen::Matrix<stan::math::var_value<double, void>, -1, 1, 0, -1, 1>&>::recursive_reducer; Partitioner = const tbb::auto_partitioner]'
\end{verbatim}

\begin{verbatim}
## stan/lib/stan_math/lib/tbb_2019_U8/include/tbb/parallel_reduce.h:178:11:   required from here
\end{verbatim}

\begin{verbatim}
## stan/lib/stan_math/stan/math/rev/functor/reduce_sum.hpp:57:23: warning: 'stan::math::internal::reduce_sum_impl<file206c10244950_model_namespace::partial_log_lik_rsfunctor__, void, stan::math::var_value<double>, const std::vector<int>&, const Eigen::Matrix<double, -1, 1, 0, -1, 1>&, const Eigen::Matrix<double, -1, -1, 0, -1, -1>&, Eigen::Matrix<stan::math::var_value<double, void>, -1, 1, 0, -1, 1>&, stan::math::var_value<double, void>&, stan::math::var_value<double, void>&, const std::vector<int, std::allocator<int> >&, const Eigen::Matrix<double, -1, 1, 0, -1, 1>&, Eigen::Matrix<stan::math::var_value<double, void>, -1, 1, 0, -1, 1>&>::recursive_reducer::local_args_tuple_scope_' will be initialized after [-Wreorder]
\end{verbatim}

\begin{verbatim}
##      scoped_args_tuple local_args_tuple_scope_;
\end{verbatim}

\begin{verbatim}
##                        ^~~~~~~~~~~~~~~~~~~~~~~
\end{verbatim}

\begin{verbatim}
## stan/lib/stan_math/stan/math/rev/functor/reduce_sum.hpp:56:25: warning:   'std::tuple<const Eigen::Matrix<double, -1, 1, 0, -1, 1>&, const Eigen::Matrix<double, -1, -1, 0, -1, -1>&, Eigen::Matrix<stan::math::var_value<double, void>, -1, 1, 0, -1, 1>&, stan::math::var_value<double, void>&, stan::math::var_value<double, void>&, const std::vector<int, std::allocator<int> >&, const Eigen::Matrix<double, -1, 1, 0, -1, 1>&, Eigen::Matrix<stan::math::var_value<double, void>, -1, 1, 0, -1, 1>&> stan::math::internal::reduce_sum_impl<file206c10244950_model_namespace::partial_log_lik_rsfunctor__, void, stan::math::var_value<double>, const std::vector<int>&, const Eigen::Matrix<double, -1, 1, 0, -1, 1>&, const Eigen::Matrix<double, -1, -1, 0, -1, -1>&, Eigen::Matrix<stan::math::var_value<double, void>, -1, 1, 0, -1, 1>&, stan::math::var_value<double, void>&, stan::math::var_value<double, void>&, const std::vector<int, std::allocator<int> >&, const Eigen::Matrix<double, -1, 1, 0, -1, 1>&, Eigen::Matrix<stan::math::var_value<double, void>, -1, 1, 0, -1, 1>&>::recursive_reducer::args_tuple_' [-Wreorder]
\end{verbatim}

\begin{verbatim}
##      std::tuple<Args...> args_tuple_;
\end{verbatim}

\begin{verbatim}
##                          ^~~~~~~~~~~
\end{verbatim}

\begin{verbatim}
## stan/lib/stan_math/stan/math/rev/functor/reduce_sum.hpp:77:5: warning:   when initialized here [-Wreorder]
\end{verbatim}

\begin{verbatim}
##      recursive_reducer(recursive_reducer& other, tbb::split)
\end{verbatim}

\begin{verbatim}
##      ^~~~~~~~~~~~~~~~~
\end{verbatim}

\begin{verbatim}
## Start sampling
\end{verbatim}

\begin{verbatim}
## Running MCMC with 2 parallel chains, with 2 thread(s) per chain...
## 
## Chain 1 Iteration:    1 / 2000 [  0%]  (Warmup) 
## Chain 1 Iteration:  100 / 2000 [  5%]  (Warmup) 
## Chain 1 Iteration:  200 / 2000 [ 10%]  (Warmup) 
## Chain 2 Iteration:    1 / 2000 [  0%]  (Warmup) 
## Chain 2 Iteration:  100 / 2000 [  5%]  (Warmup) 
## Chain 1 Iteration:  300 / 2000 [ 15%]  (Warmup) 
## Chain 1 Iteration:  400 / 2000 [ 20%]  (Warmup) 
## Chain 1 Iteration:  500 / 2000 [ 25%]  (Warmup) 
## Chain 1 Iteration:  600 / 2000 [ 30%]  (Warmup) 
## Chain 1 Iteration:  700 / 2000 [ 35%]  (Warmup) 
## Chain 1 Iteration:  800 / 2000 [ 40%]  (Warmup) 
## Chain 1 Iteration:  900 / 2000 [ 45%]  (Warmup) 
## Chain 1 Iteration: 1000 / 2000 [ 50%]  (Warmup) 
## Chain 1 Iteration: 1001 / 2000 [ 50%]  (Sampling) 
## Chain 2 Iteration:  200 / 2000 [ 10%]  (Warmup) 
## Chain 2 Iteration:  300 / 2000 [ 15%]  (Warmup) 
## Chain 2 Iteration:  400 / 2000 [ 20%]  (Warmup) 
## Chain 2 Iteration:  500 / 2000 [ 25%]  (Warmup) 
## Chain 2 Iteration:  600 / 2000 [ 30%]  (Warmup) 
## Chain 2 Iteration:  700 / 2000 [ 35%]  (Warmup) 
## Chain 2 Iteration:  800 / 2000 [ 40%]  (Warmup) 
## Chain 2 Iteration:  900 / 2000 [ 45%]  (Warmup) 
## Chain 1 Iteration: 1100 / 2000 [ 55%]  (Sampling) 
## Chain 1 Iteration: 1200 / 2000 [ 60%]  (Sampling) 
## Chain 2 Iteration: 1000 / 2000 [ 50%]  (Warmup) 
## Chain 2 Iteration: 1001 / 2000 [ 50%]  (Sampling) 
## Chain 2 Iteration: 1100 / 2000 [ 55%]  (Sampling) 
## Chain 2 Iteration: 1200 / 2000 [ 60%]  (Sampling) 
## Chain 1 Iteration: 1300 / 2000 [ 65%]  (Sampling) 
## Chain 1 Iteration: 1400 / 2000 [ 70%]  (Sampling) 
## Chain 1 Iteration: 1500 / 2000 [ 75%]  (Sampling) 
## Chain 2 Iteration: 1300 / 2000 [ 65%]  (Sampling) 
## Chain 2 Iteration: 1400 / 2000 [ 70%]  (Sampling) 
## Chain 1 Iteration: 1600 / 2000 [ 80%]  (Sampling) 
## Chain 1 Iteration: 1700 / 2000 [ 85%]  (Sampling) 
## Chain 2 Iteration: 1500 / 2000 [ 75%]  (Sampling) 
## Chain 2 Iteration: 1600 / 2000 [ 80%]  (Sampling) 
## Chain 2 Iteration: 1700 / 2000 [ 85%]  (Sampling) 
## Chain 1 Iteration: 1800 / 2000 [ 90%]  (Sampling) 
## Chain 1 Iteration: 1900 / 2000 [ 95%]  (Sampling) 
## Chain 1 Iteration: 2000 / 2000 [100%]  (Sampling) 
## Chain 2 Iteration: 1800 / 2000 [ 90%]  (Sampling) 
## Chain 2 Iteration: 1900 / 2000 [ 95%]  (Sampling) 
## Chain 1 finished in 0.7 seconds.
## Chain 2 Iteration: 2000 / 2000 [100%]  (Sampling) 
## Chain 2 finished in 0.7 seconds.
## 
## Both chains finished successfully.
## Mean chain execution time: 0.7 seconds.
## Total execution time: 0.9 seconds.
\end{verbatim}

\begin{Shaded}
\begin{Highlighting}[]
\CommentTok{\# prior predictive check}
\FunctionTok{pp\_check}\NormalTok{(iqr\_meta\_spo, }\AttributeTok{nsamples =} \DecValTok{100}\NormalTok{)}
\end{Highlighting}
\end{Shaded}

\includegraphics{Assignment3-final_files/figure-latex/unnamed-chunk-7-1.pdf}

\begin{Shaded}
\begin{Highlighting}[]
\CommentTok{\# run model now with data}
\NormalTok{iqr\_meta\_spt }\OtherTok{\textless{}{-}} \FunctionTok{brm}\NormalTok{(}
  \AttributeTok{formula =}\NormalTok{ iqr\_m0,}
  \AttributeTok{data =}\NormalTok{ data,}
  \AttributeTok{family =}\NormalTok{ gaussian,}
  \AttributeTok{prior =}\NormalTok{ meta\_pri,}
  \AttributeTok{sample\_prior =}\NormalTok{ T,}
  \AttributeTok{chains =} \DecValTok{2}\NormalTok{,}
  \AttributeTok{cores =} \DecValTok{2}\NormalTok{,}
  \AttributeTok{backend =} \StringTok{"cmdstanr"}\NormalTok{,}
  \AttributeTok{threads =} \FunctionTok{threading}\NormalTok{(}\DecValTok{2}\NormalTok{)}
\NormalTok{)}
\end{Highlighting}
\end{Shaded}

\begin{verbatim}
## Warning: Rows containing NAs were excluded from the model.
\end{verbatim}

\begin{verbatim}
## Compiling Stan program...
\end{verbatim}

\begin{verbatim}
## In file included from stan/lib/stan_math/stan/math/rev/functor.hpp:28,
\end{verbatim}

\begin{verbatim}
##                  from stan/lib/stan_math/stan/math/rev.hpp:11,
\end{verbatim}

\begin{verbatim}
##                  from stan/lib/stan_math/stan/math.hpp:19,
\end{verbatim}

\begin{verbatim}
##                  from stan/src/stan/model/model_header.hpp:4,
\end{verbatim}

\begin{verbatim}
##                  from C:/Users/ukoto/AppData/Local/Temp/RtmpCcFj0B/model-206c507042e5.hpp:3:
\end{verbatim}

\begin{verbatim}
## stan/lib/stan_math/stan/math/rev/functor/reduce_sum.hpp: In instantiation of 'stan::math::internal::reduce_sum_impl<ReduceFunction, typename std::enable_if<stan::is_var<typename std::decay<_Arg>::type, void>::value, void>::type, ReturnType, Vec, Args ...>::recursive_reducer::recursive_reducer(size_t, size_t, double*, VecT&&, ArgsT&& ...) [with VecT = const std::vector<int>&; ArgsT = {const Eigen::Matrix<double, -1, 1, 0, -1, 1>&, const Eigen::Matrix<double, -1, -1, 0, -1, -1>&, Eigen::Matrix<stan::math::var_value<double, void>, -1, 1, 0, -1, 1>&, stan::math::var_value<double, void>&, stan::math::var_value<double, void>&, const std::vector<int, std::allocator<int> >&, const Eigen::Matrix<double, -1, 1, 0, -1, 1>&, Eigen::Matrix<stan::math::var_value<double, void>, -1, 1, 0, -1, 1>&}; ReduceFunction = file206c4a767a20_model_namespace::partial_log_lik_rsfunctor__; ReturnType = stan::math::var_value<double>; Vec = const std::vector<int>&; Args = {const Eigen::Matrix<double, -1, 1, 0, -1, 1>&, const Eigen::Matrix<double, -1, -1, 0, -1, -1>&, Eigen::Matrix<stan::math::var_value<double, void>, -1, 1, 0, -1, 1>&, stan::math::var_value<double, void>&, stan::math::var_value<double, void>&, const std::vector<int, std::allocator<int> >&, const Eigen::Matrix<double, -1, 1, 0, -1, 1>&, Eigen::Matrix<stan::math::var_value<double, void>, -1, 1, 0, -1, 1>&}; size_t = long long unsigned int]':
\end{verbatim}

\begin{verbatim}
## stan/lib/stan_math/stan/math/rev/functor/reduce_sum.hpp:248:23:   required from 'stan::math::var stan::math::internal::reduce_sum_impl<ReduceFunction, typename std::enable_if<stan::is_var<typename std::decay<_Arg>::type, void>::value, void>::type, ReturnType, Vec, Args ...>::operator()(Vec&&, bool, int, std::ostream*, Args&& ...) const [with ReduceFunction = file206c4a767a20_model_namespace::partial_log_lik_rsfunctor__; ReturnType = stan::math::var_value<double>; Vec = const std::vector<int>&; Args = {const Eigen::Matrix<double, -1, 1, 0, -1, 1>&, const Eigen::Matrix<double, -1, -1, 0, -1, -1>&, Eigen::Matrix<stan::math::var_value<double, void>, -1, 1, 0, -1, 1>&, stan::math::var_value<double, void>&, stan::math::var_value<double, void>&, const std::vector<int, std::allocator<int> >&, const Eigen::Matrix<double, -1, 1, 0, -1, 1>&, Eigen::Matrix<stan::math::var_value<double, void>, -1, 1, 0, -1, 1>&}; typename std::enable_if<stan::is_var<typename std::decay<_Arg>::type, void>::value, void>::type = void; stan::math::var = stan::math::var_value<double>; std::ostream = std::basic_ostream<char>]'
\end{verbatim}

\begin{verbatim}
## stan/lib/stan_math/stan/math/prim/functor/reduce_sum.hpp:207:46:   required from 'auto stan::math::reduce_sum(Vec&&, int, std::ostream*, Args&& ...) [with ReduceFunction = file206c4a767a20_model_namespace::partial_log_lik_rsfunctor__; Vec = const std::vector<int>&; <template-parameter-1-3> = void; Args = {const Eigen::Matrix<double, -1, 1, 0, -1, 1>&, const Eigen::Matrix<double, -1, -1, 0, -1, -1>&, Eigen::Matrix<stan::math::var_value<double, void>, -1, 1, 0, -1, 1>&, stan::math::var_value<double, void>&, stan::math::var_value<double, void>&, const std::vector<int, std::allocator<int> >&, const Eigen::Matrix<double, -1, 1, 0, -1, 1>&, Eigen::Matrix<stan::math::var_value<double, void>, -1, 1, 0, -1, 1>&}; std::ostream = std::basic_ostream<char>]'
\end{verbatim}

\begin{verbatim}
## C:/Users/ukoto/AppData/Local/Temp/RtmpCcFj0B/model-206c507042e5.hpp:637:52:   required from 'stan::scalar_type_t<T2> file206c4a767a20_model_namespace::file206c4a767a20_model::log_prob_impl(VecR&, VecI&, std::ostream*) const [with bool propto__ = false; bool jacobian__ = false; VecR = Eigen::Matrix<stan::math::var_value<double>, -1, 1>; VecI = Eigen::Matrix<int, -1, 1>; stan::require_vector_like_t<VecR>* <anonymous> = 0; stan::require_vector_like_vt<std::is_integral, VecI>* <anonymous> = 0; stan::scalar_type_t<T2> = stan::math::var_value<double>; std::ostream = std::basic_ostream<char>]'
\end{verbatim}

\begin{verbatim}
## C:/Users/ukoto/AppData/Local/Temp/RtmpCcFj0B/model-206c507042e5.hpp:1081:77:   required from 'T_ file206c4a767a20_model_namespace::file206c4a767a20_model::log_prob(Eigen::Matrix<T_job_param, -1, 1>&, std::ostream*) const [with bool propto__ = false; bool jacobian__ = false; T_ = stan::math::var_value<double>; std::ostream = std::basic_ostream<char>]'
\end{verbatim}

\begin{verbatim}
## stan/src/stan/model/model_base_crtp.hpp:96:77:   required from 'stan::math::var stan::model::model_base_crtp<M>::log_prob(Eigen::Matrix<stan::math::var_value<double>, -1, 1>&, std::ostream*) const [with M = file206c4a767a20_model_namespace::file206c4a767a20_model; stan::math::var = stan::math::var_value<double>; std::ostream = std::basic_ostream<char>]'
\end{verbatim}

\begin{verbatim}
## stan/src/stan/model/model_base_crtp.hpp:93:20:   required from here
\end{verbatim}

\begin{verbatim}
## stan/lib/stan_math/stan/math/rev/functor/reduce_sum.hpp:57:23: warning: 'stan::math::internal::reduce_sum_impl<file206c4a767a20_model_namespace::partial_log_lik_rsfunctor__, void, stan::math::var_value<double>, const std::vector<int>&, const Eigen::Matrix<double, -1, 1, 0, -1, 1>&, const Eigen::Matrix<double, -1, -1, 0, -1, -1>&, Eigen::Matrix<stan::math::var_value<double, void>, -1, 1, 0, -1, 1>&, stan::math::var_value<double, void>&, stan::math::var_value<double, void>&, const std::vector<int, std::allocator<int> >&, const Eigen::Matrix<double, -1, 1, 0, -1, 1>&, Eigen::Matrix<stan::math::var_value<double, void>, -1, 1, 0, -1, 1>&>::recursive_reducer::local_args_tuple_scope_' will be initialized after [-Wreorder]
\end{verbatim}

\begin{verbatim}
##      scoped_args_tuple local_args_tuple_scope_;
\end{verbatim}

\begin{verbatim}
##                        ^~~~~~~~~~~~~~~~~~~~~~~
\end{verbatim}

\begin{verbatim}
## stan/lib/stan_math/stan/math/rev/functor/reduce_sum.hpp:56:25: warning:   'std::tuple<const Eigen::Matrix<double, -1, 1, 0, -1, 1>&, const Eigen::Matrix<double, -1, -1, 0, -1, -1>&, Eigen::Matrix<stan::math::var_value<double, void>, -1, 1, 0, -1, 1>&, stan::math::var_value<double, void>&, stan::math::var_value<double, void>&, const std::vector<int, std::allocator<int> >&, const Eigen::Matrix<double, -1, 1, 0, -1, 1>&, Eigen::Matrix<stan::math::var_value<double, void>, -1, 1, 0, -1, 1>&> stan::math::internal::reduce_sum_impl<file206c4a767a20_model_namespace::partial_log_lik_rsfunctor__, void, stan::math::var_value<double>, const std::vector<int>&, const Eigen::Matrix<double, -1, 1, 0, -1, 1>&, const Eigen::Matrix<double, -1, -1, 0, -1, -1>&, Eigen::Matrix<stan::math::var_value<double, void>, -1, 1, 0, -1, 1>&, stan::math::var_value<double, void>&, stan::math::var_value<double, void>&, const std::vector<int, std::allocator<int> >&, const Eigen::Matrix<double, -1, 1, 0, -1, 1>&, Eigen::Matrix<stan::math::var_value<double, void>, -1, 1, 0, -1, 1>&>::recursive_reducer::args_tuple_' [-Wreorder]
\end{verbatim}

\begin{verbatim}
##      std::tuple<Args...> args_tuple_;
\end{verbatim}

\begin{verbatim}
##                          ^~~~~~~~~~~
\end{verbatim}

\begin{verbatim}
## stan/lib/stan_math/stan/math/rev/functor/reduce_sum.hpp:62:5: warning:   when initialized here [-Wreorder]
\end{verbatim}

\begin{verbatim}
##      recursive_reducer(size_t num_vars_per_term, size_t num_vars_shared_terms,
\end{verbatim}

\begin{verbatim}
##      ^~~~~~~~~~~~~~~~~
\end{verbatim}

\begin{verbatim}
## stan/lib/stan_math/stan/math/rev/functor/reduce_sum.hpp: In instantiation of 'stan::math::internal::reduce_sum_impl<ReduceFunction, typename std::enable_if<stan::is_var<typename std::decay<_Arg>::type, void>::value, void>::type, ReturnType, Vec, Args ...>::recursive_reducer::recursive_reducer(stan::math::internal::reduce_sum_impl<ReduceFunction, typename std::enable_if<stan::is_var<typename std::decay<_Arg>::type, void>::value, void>::type, ReturnType, Vec, Args ...>::recursive_reducer&, tbb::split) [with ReduceFunction = file206c4a767a20_model_namespace::partial_log_lik_rsfunctor__; ReturnType = stan::math::var_value<double>; Vec = const std::vector<int>&; Args = {const Eigen::Matrix<double, -1, 1, 0, -1, 1>&, const Eigen::Matrix<double, -1, -1, 0, -1, -1>&, Eigen::Matrix<stan::math::var_value<double, void>, -1, 1, 0, -1, 1>&, stan::math::var_value<double, void>&, stan::math::var_value<double, void>&, const std::vector<int, std::allocator<int> >&, const Eigen::Matrix<double, -1, 1, 0, -1, 1>&, Eigen::Matrix<stan::math::var_value<double, void>, -1, 1, 0, -1, 1>&}]':
\end{verbatim}

\begin{verbatim}
## stan/lib/stan_math/lib/tbb_2019_U8/include/tbb/parallel_reduce.h:183:27:   required from 'tbb::task* tbb::interface9::internal::start_reduce<Range, Body, Partitioner>::execute() [with Range = tbb::blocked_range<long long unsigned int>; Body = stan::math::internal::reduce_sum_impl<file206c4a767a20_model_namespace::partial_log_lik_rsfunctor__, void, stan::math::var_value<double>, const std::vector<int>&, const Eigen::Matrix<double, -1, 1, 0, -1, 1>&, const Eigen::Matrix<double, -1, -1, 0, -1, -1>&, Eigen::Matrix<stan::math::var_value<double, void>, -1, 1, 0, -1, 1>&, stan::math::var_value<double, void>&, stan::math::var_value<double, void>&, const std::vector<int, std::allocator<int> >&, const Eigen::Matrix<double, -1, 1, 0, -1, 1>&, Eigen::Matrix<stan::math::var_value<double, void>, -1, 1, 0, -1, 1>&>::recursive_reducer; Partitioner = const tbb::auto_partitioner]'
\end{verbatim}

\begin{verbatim}
## stan/lib/stan_math/lib/tbb_2019_U8/include/tbb/parallel_reduce.h:178:11:   required from here
\end{verbatim}

\begin{verbatim}
## stan/lib/stan_math/stan/math/rev/functor/reduce_sum.hpp:57:23: warning: 'stan::math::internal::reduce_sum_impl<file206c4a767a20_model_namespace::partial_log_lik_rsfunctor__, void, stan::math::var_value<double>, const std::vector<int>&, const Eigen::Matrix<double, -1, 1, 0, -1, 1>&, const Eigen::Matrix<double, -1, -1, 0, -1, -1>&, Eigen::Matrix<stan::math::var_value<double, void>, -1, 1, 0, -1, 1>&, stan::math::var_value<double, void>&, stan::math::var_value<double, void>&, const std::vector<int, std::allocator<int> >&, const Eigen::Matrix<double, -1, 1, 0, -1, 1>&, Eigen::Matrix<stan::math::var_value<double, void>, -1, 1, 0, -1, 1>&>::recursive_reducer::local_args_tuple_scope_' will be initialized after [-Wreorder]
\end{verbatim}

\begin{verbatim}
##      scoped_args_tuple local_args_tuple_scope_;
\end{verbatim}

\begin{verbatim}
##                        ^~~~~~~~~~~~~~~~~~~~~~~
\end{verbatim}

\begin{verbatim}
## stan/lib/stan_math/stan/math/rev/functor/reduce_sum.hpp:56:25: warning:   'std::tuple<const Eigen::Matrix<double, -1, 1, 0, -1, 1>&, const Eigen::Matrix<double, -1, -1, 0, -1, -1>&, Eigen::Matrix<stan::math::var_value<double, void>, -1, 1, 0, -1, 1>&, stan::math::var_value<double, void>&, stan::math::var_value<double, void>&, const std::vector<int, std::allocator<int> >&, const Eigen::Matrix<double, -1, 1, 0, -1, 1>&, Eigen::Matrix<stan::math::var_value<double, void>, -1, 1, 0, -1, 1>&> stan::math::internal::reduce_sum_impl<file206c4a767a20_model_namespace::partial_log_lik_rsfunctor__, void, stan::math::var_value<double>, const std::vector<int>&, const Eigen::Matrix<double, -1, 1, 0, -1, 1>&, const Eigen::Matrix<double, -1, -1, 0, -1, -1>&, Eigen::Matrix<stan::math::var_value<double, void>, -1, 1, 0, -1, 1>&, stan::math::var_value<double, void>&, stan::math::var_value<double, void>&, const std::vector<int, std::allocator<int> >&, const Eigen::Matrix<double, -1, 1, 0, -1, 1>&, Eigen::Matrix<stan::math::var_value<double, void>, -1, 1, 0, -1, 1>&>::recursive_reducer::args_tuple_' [-Wreorder]
\end{verbatim}

\begin{verbatim}
##      std::tuple<Args...> args_tuple_;
\end{verbatim}

\begin{verbatim}
##                          ^~~~~~~~~~~
\end{verbatim}

\begin{verbatim}
## stan/lib/stan_math/stan/math/rev/functor/reduce_sum.hpp:77:5: warning:   when initialized here [-Wreorder]
\end{verbatim}

\begin{verbatim}
##      recursive_reducer(recursive_reducer& other, tbb::split)
\end{verbatim}

\begin{verbatim}
##      ^~~~~~~~~~~~~~~~~
\end{verbatim}

\begin{verbatim}
## Start sampling
\end{verbatim}

\begin{verbatim}
## Running MCMC with 2 parallel chains, with 2 thread(s) per chain...
## 
## Chain 1 Iteration:    1 / 2000 [  0%]  (Warmup) 
## Chain 2 Iteration:    1 / 2000 [  0%]  (Warmup) 
## Chain 1 Iteration:  100 / 2000 [  5%]  (Warmup) 
## Chain 2 Iteration:  100 / 2000 [  5%]  (Warmup) 
## Chain 1 Iteration:  200 / 2000 [ 10%]  (Warmup) 
## Chain 2 Iteration:  200 / 2000 [ 10%]  (Warmup) 
## Chain 1 Iteration:  300 / 2000 [ 15%]  (Warmup) 
## Chain 1 Iteration:  400 / 2000 [ 20%]  (Warmup) 
## Chain 2 Iteration:  300 / 2000 [ 15%]  (Warmup) 
## Chain 1 Iteration:  500 / 2000 [ 25%]  (Warmup) 
## Chain 2 Iteration:  400 / 2000 [ 20%]  (Warmup) 
## Chain 2 Iteration:  500 / 2000 [ 25%]  (Warmup) 
## Chain 1 Iteration:  600 / 2000 [ 30%]  (Warmup) 
## Chain 2 Iteration:  600 / 2000 [ 30%]  (Warmup) 
## Chain 1 Iteration:  700 / 2000 [ 35%]  (Warmup) 
## Chain 1 Iteration:  800 / 2000 [ 40%]  (Warmup) 
## Chain 2 Iteration:  700 / 2000 [ 35%]  (Warmup) 
## Chain 2 Iteration:  800 / 2000 [ 40%]  (Warmup) 
## Chain 1 Iteration:  900 / 2000 [ 45%]  (Warmup) 
## Chain 2 Iteration:  900 / 2000 [ 45%]  (Warmup) 
## Chain 1 Iteration: 1000 / 2000 [ 50%]  (Warmup) 
## Chain 1 Iteration: 1001 / 2000 [ 50%]  (Sampling) 
## Chain 2 Iteration: 1000 / 2000 [ 50%]  (Warmup) 
## Chain 2 Iteration: 1001 / 2000 [ 50%]  (Sampling) 
## Chain 1 Iteration: 1100 / 2000 [ 55%]  (Sampling) 
## Chain 2 Iteration: 1100 / 2000 [ 55%]  (Sampling) 
## Chain 1 Iteration: 1200 / 2000 [ 60%]  (Sampling) 
## Chain 2 Iteration: 1200 / 2000 [ 60%]  (Sampling) 
## Chain 1 Iteration: 1300 / 2000 [ 65%]  (Sampling) 
## Chain 2 Iteration: 1300 / 2000 [ 65%]  (Sampling) 
## Chain 1 Iteration: 1400 / 2000 [ 70%]  (Sampling) 
## Chain 2 Iteration: 1400 / 2000 [ 70%]  (Sampling) 
## Chain 1 Iteration: 1500 / 2000 [ 75%]  (Sampling) 
## Chain 2 Iteration: 1500 / 2000 [ 75%]  (Sampling) 
## Chain 1 Iteration: 1600 / 2000 [ 80%]  (Sampling) 
## Chain 2 Iteration: 1600 / 2000 [ 80%]  (Sampling) 
## Chain 1 Iteration: 1700 / 2000 [ 85%]  (Sampling) 
## Chain 1 Iteration: 1800 / 2000 [ 90%]  (Sampling) 
## Chain 2 Iteration: 1700 / 2000 [ 85%]  (Sampling) 
## Chain 1 Iteration: 1900 / 2000 [ 95%]  (Sampling) 
## Chain 2 Iteration: 1800 / 2000 [ 90%]  (Sampling) 
## Chain 1 Iteration: 2000 / 2000 [100%]  (Sampling) 
## Chain 2 Iteration: 1900 / 2000 [ 95%]  (Sampling) 
## Chain 1 finished in 2.3 seconds.
## Chain 2 Iteration: 2000 / 2000 [100%]  (Sampling) 
## Chain 2 finished in 2.4 seconds.
## 
## Both chains finished successfully.
## Mean chain execution time: 2.4 seconds.
## Total execution time: 2.6 seconds.
\end{verbatim}

\begin{Shaded}
\begin{Highlighting}[]
\CommentTok{\# posterior predictive check}
\FunctionTok{pp\_check}\NormalTok{(iqr\_meta\_spt, }\AttributeTok{nsamples =} \DecValTok{100}\NormalTok{)}
\end{Highlighting}
\end{Shaded}

\includegraphics{Assignment3-final_files/figure-latex/unnamed-chunk-7-2.pdf}

\begin{Shaded}
\begin{Highlighting}[]
\CommentTok{\# check effect of diagnosis}
\FunctionTok{hypothesis}\NormalTok{(iqr\_meta\_spt, }\StringTok{"DiagnosisTD \textless{} 0"}\NormalTok{)}
\end{Highlighting}
\end{Shaded}

\begin{verbatim}
## Hypothesis Tests for class b:
##          Hypothesis Estimate Est.Error CI.Lower CI.Upper Evid.Ratio Post.Prob
## 1 (DiagnosisTD) < 0    -0.36      0.07    -0.47    -0.24        Inf         1
##   Star
## 1    *
## ---
## 'CI': 90%-CI for one-sided and 95%-CI for two-sided hypotheses.
## '*': For one-sided hypotheses, the posterior probability exceeds 95%;
## for two-sided hypotheses, the value tested against lies outside the 95%-CI.
## Posterior probabilities of point hypotheses assume equal prior probabilities.
\end{verbatim}

\begin{Shaded}
\begin{Highlighting}[]
\FunctionTok{plot}\NormalTok{(}\FunctionTok{hypothesis}\NormalTok{(iqr\_meta\_spt, }\StringTok{"DiagnosisTD \textless{} 0"}\NormalTok{))}
\end{Highlighting}
\end{Shaded}

\includegraphics{Assignment3-final_files/figure-latex/unnamed-chunk-7-3.pdf}

\begin{Shaded}
\begin{Highlighting}[]
\CommentTok{\# summarise model}
\FunctionTok{summary}\NormalTok{(iqr\_meta\_spt)}
\end{Highlighting}
\end{Shaded}

\begin{verbatim}
##  Family: gaussian 
##   Links: mu = identity; sigma = identity 
## Formula: Pitch_IQR_stan ~ 1 + Diagnosis + (1 | ID) 
##    Data: data (Number of observations: 1042) 
## Samples: 2 chains, each with iter = 2000; warmup = 1000; thin = 1;
##          total post-warmup samples = 2000
## 
## Group-Level Effects: 
## ~ID (Number of levels: 140) 
##               Estimate Est.Error l-95% CI u-95% CI Rhat Bulk_ESS Tail_ESS
## sd(Intercept)     0.63      0.04     0.56     0.70 1.00      864     1264
## 
## Population-Level Effects: 
##             Estimate Est.Error l-95% CI u-95% CI Rhat Bulk_ESS Tail_ESS
## Intercept       0.32      0.06     0.20     0.44 1.00      866     1187
## DiagnosisTD    -0.36      0.07    -0.50    -0.22 1.00     1196     1356
## 
## Family Specific Parameters: 
##       Estimate Est.Error l-95% CI u-95% CI Rhat Bulk_ESS Tail_ESS
## sigma     0.71      0.02     0.68     0.75 1.00     2555     1084
## 
## Samples were drawn using sample(hmc). For each parameter, Bulk_ESS
## and Tail_ESS are effective sample size measures, and Rhat is the potential
## scale reduction factor on split chains (at convergence, Rhat = 1).
\end{verbatim}

\begin{Shaded}
\begin{Highlighting}[]
\CommentTok{\# plot model}
\FunctionTok{plot}\NormalTok{(iqr\_meta\_spt)}
\end{Highlighting}
\end{Shaded}

\includegraphics{Assignment3-final_files/figure-latex/unnamed-chunk-7-4.pdf}

\#\#\#\#Step 5: Compare the models - Plot priors and posteriors of the
diagnosis effect in both models - Compare posteriors between the two
models - Compare the two models (LOO) - Discuss how they compare and
whether any of them is best.

\begin{Shaded}
\begin{Highlighting}[]
\CommentTok{\#magnus}
\CommentTok{\# plot priors and posteriors for the two models}
\FunctionTok{plot}\NormalTok{(}\FunctionTok{hypothesis}\NormalTok{(iqr\_m0\_spt, }\StringTok{\textquotesingle{}DiagnosisTD \textless{} 0\textquotesingle{}}\NormalTok{))}
\end{Highlighting}
\end{Shaded}

\includegraphics{Assignment3-final_files/figure-latex/unnamed-chunk-8-1.pdf}

\begin{Shaded}
\begin{Highlighting}[]
\FunctionTok{plot}\NormalTok{(}\FunctionTok{hypothesis}\NormalTok{(iqr\_meta\_spt, }\StringTok{\textquotesingle{}DiagnosisTD \textless{} 0\textquotesingle{}}\NormalTok{))}
\end{Highlighting}
\end{Shaded}

\includegraphics{Assignment3-final_files/figure-latex/unnamed-chunk-8-2.pdf}

\begin{Shaded}
\begin{Highlighting}[]
\CommentTok{\# checking posteriors for the models}
\FunctionTok{pp\_check}\NormalTok{(iqr\_m0\_spt)}
\end{Highlighting}
\end{Shaded}

\begin{verbatim}
## Using 10 posterior samples for ppc type 'dens_overlay' by default.
\end{verbatim}

\includegraphics{Assignment3-final_files/figure-latex/unnamed-chunk-8-3.pdf}

\begin{Shaded}
\begin{Highlighting}[]
\FunctionTok{pp\_check}\NormalTok{(iqr\_meta\_spt)}
\end{Highlighting}
\end{Shaded}

\begin{verbatim}
## Using 10 posterior samples for ppc type 'dens_overlay' by default.
\end{verbatim}

\includegraphics{Assignment3-final_files/figure-latex/unnamed-chunk-8-4.pdf}

\begin{Shaded}
\begin{Highlighting}[]
\CommentTok{\#loo comparison}
\FunctionTok{loo}\NormalTok{(iqr\_m0\_spt, iqr\_meta\_spt)}
\end{Highlighting}
\end{Shaded}

\begin{verbatim}
## Warning: Found 5 observations with a pareto_k > 0.7 in model 'iqr_m0_spt'. It is
## recommended to set 'moment_match = TRUE' in order to perform moment matching for
## problematic observations.
\end{verbatim}

\begin{verbatim}
## Warning: Found 7 observations with a pareto_k > 0.7 in model 'iqr_meta_spt'. It
## is recommended to set 'moment_match = TRUE' in order to perform moment matching
## for problematic observations.
\end{verbatim}

\begin{verbatim}
## Output of model 'iqr_m0_spt':
## 
## Computed from 2000 by 1042 log-likelihood matrix
## 
##          Estimate   SE
## elpd_loo  -1182.5 44.7
## p_loo       103.0  9.7
## looic      2365.0 89.4
## ------
## Monte Carlo SE of elpd_loo is NA.
## 
## Pareto k diagnostic values:
##                          Count Pct.    Min. n_eff
## (-Inf, 0.5]   (good)     1027  98.6%   220       
##  (0.5, 0.7]   (ok)         10   1.0%   82        
##    (0.7, 1]   (bad)         4   0.4%   25        
##    (1, Inf)   (very bad)    1   0.1%   19        
## See help('pareto-k-diagnostic') for details.
## 
## Output of model 'iqr_meta_spt':
## 
## Computed from 2000 by 1042 log-likelihood matrix
## 
##          Estimate   SE
## elpd_loo  -1183.4 45.0
## p_loo       104.4 10.4
## looic      2366.9 90.0
## ------
## Monte Carlo SE of elpd_loo is NA.
## 
## Pareto k diagnostic values:
##                          Count Pct.    Min. n_eff
## (-Inf, 0.5]   (good)     1027  98.6%   244       
##  (0.5, 0.7]   (ok)          8   0.8%   106       
##    (0.7, 1]   (bad)         6   0.6%   34        
##    (1, Inf)   (very bad)    1   0.1%   3         
## See help('pareto-k-diagnostic') for details.
## 
## Model comparisons:
##              elpd_diff se_diff
## iqr_m0_spt    0.0       0.0   
## iqr_meta_spt -0.9       1.6
\end{verbatim}

\#\#\#\#Step 6: Prepare a nice write up of the analysis and answer the
questions at the top.

Optional step 7: how skeptical should a prior be? - Try different levels
of skepticism and compare them both plotting the impact on the inferred
effect size and using LOO.

Optional step 8: Include other predictors - Do age, gender and education
improve the model? - Should they be main effects or interactions?

Optional step 9: generalized linear models - If you wanted to preserve
the lognormal distribution of the pitch variability, what would

snscrape --jsonl --progress --max-results 500 --since 2021-03-08
twitter-search ``\#AuratMarch ban OR hang OR blasphemous
until:2021-03-12'' \textgreater{} text-query-tweets.json

\end{document}
